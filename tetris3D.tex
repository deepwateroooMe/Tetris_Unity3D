% Created 2019-09-28 Sat 20:01
\documentclass[9pt, b5paper]{article}
\usepackage{fontspec}
\usepackage{graphicx}
\usepackage{xcolor}
\usepackage[slantfont,boldfont]{xeCJK}
\setCJKmainfont[BoldFont = Heiti SC, ItalicFont = STFangsong]{STSong}
\setCJKsansfont{STHeiti}
\setCJKmonofont{STFangsong}
\usepackage{multirow}
\usepackage{multicol}
\usepackage{float}
\usepackage{textcomp}
\usepackage{geometry}
\geometry{left=1.2cm,right=1.2cm,top=1.5cm,bottom=1.0cm}
\usepackage{algorithm}
\usepackage{algorithmic}
\usepackage{latexsym}
\usepackage{natbib}
\usepackage{listings}
\usepackage{minted}
\usepackage[xetex,colorlinks=true,CJKbookmarks=true,linkcolor=blue,urlcolor=blue,menucolor=blue]{hyperref}
\author{deepwaterooo}
\date{\today}
\title{Tetris 3D Reference}
\hypersetup{
  pdfkeywords={},
  pdfsubject={},
  pdfcreator={Emacs 25.3.1 (Org mode 8.2.7c)}}
\begin{document}

\maketitle
\tableofcontents


\section{ShaderS基础知识总结}
\label{sec-1}
\begin{itemize}
\item Unity3d中Shader的一些关于矩阵变换的基本信息
\begin{itemize}
\item \url{https://blog.csdn.net/yutyliu/article/details/56013807}
\end{itemize}
\item shader内置函数
\begin{itemize}
\item \url{https://blog.csdn.net/qingshui37/article/details/51476404}
\end{itemize}
\end{itemize}
\subsection{基本变换}
\label{sec-1-1}
\begin{itemize}
\item 在Unity中,每个物体都有一个坐标系,就是自身坐标系,各个物体之间相互独立。
\item 所有的物体都处在一个统一的空间里,这个空间就是世界空间,也有一个世界坐标系。
\item 把一个3D物体渲染到2D的屏幕上的基本流程以及每个变换对应的矩阵

\begin{itemize}
\item 因为物体的顶点坐标是基于自身坐标系的,所以渲染时,最先的变换是 模型空间——>世界空间,对应矩阵:\_Object2World
\item 物体要渲染到相机平面上实际上,是相机的可视区域内有哪些物体,也就是物体处于相机坐标系的本地坐标(localPosition)处于哪个位置。这个变换是 世界空间——>相机空间,对应矩阵:UNITY\_MATRIX\_V
\item 此刻获取到了物体处于相机空间的位置,要把相机空间的所有信息都渲染到2维图片上,此刻需要进行投影变换,透视相机投影变换的目的是为了把视锥体转换为立方体,转换后,视锥体近平面的右上角点变成立方体前平面的中心,把视锥体较小的部分放大,较大的部分缩小,形成最终的立方体。变换后的x坐标范围是[-1, 1],y坐标范围是[-1, 1],z坐标范围是[0, 1](OpenGL不同,z值范围是[-1, 1]),这个变换是 相机空间——>投影空间,对应矩阵:UNITY\_MATRIX\_P
\item 通过UNITY\_MATRIX\_MVP 这个矩阵,可以把物体的顶点位置从模型自身坐标系转换到投影空间。
\item 对投影矩阵感兴趣的,可以自己搜索一下,整个推导过程需要一定的数学基础,理解就行。
\end{itemize}
\end{itemize}

\subsection{顶点转换函数}
\label{sec-1-2}
\subsubsection{float4 UnityObjectToClipPos(float3 pos)}
\label{sec-1-2-1}
\begin{itemize}
\item Transforms a point from object space to the camera’s clip space in homogeneous coordinates. This is the equivalent of mul(UNITY\_MATRIX\_MVP, float4(pos, 1.0)), and should be used in its place.
\item homogeneous coordinates:齐次坐标
\item 等价于:mul(UNITY\_MATRIX\_MVP, float4(pos, 1.0)),
\end{itemize}
\subsubsection{float3 UnityObjectToViewPos(float3 pos)}
\label{sec-1-2-2}
\begin{itemize}
\item Transforms a point from object space to view space. This is the equivalent of mul(UNITY\_MATRIX\_MV, float4(pos, 1.0)).xyz, and should be used in its place.
\item 等价于:mul(UNITY\_MATRIX\_MV, float4(pos, 1.0)).
\end{itemize}

\subsection{Forwardrendering helper functions in UnityCG.cginc}
\label{sec-1-3}
\begin{itemize}
\item These functions are only useful when using forward rendering(ForwardBase or ForwardAdd pass types).
\item 仅用于前向渲染
\end{itemize}
\begin{center}
\begin{tabular}{ll}
\hline
Function: & Description:\\
\hline
float3 WorldSpaceLightDir (float4 v) & Computes world space direction (not normalized) to light,\\
 & given object space vertex position.\\
 & 参数是object space下的顶点坐标,取得world space下指向光源的方向\\
\hline
float3 ObjSpaceLightDir (float4 v) & Computes object space direction (not normalized) to light,\\
 & given object space vertex position.\\
 & 参数是object space下的顶点坐标,取得object space下指向光源的方向\\
\hline
float3 Shade4PointLights (\ldots{}) & Computes illumination from four point lights, with light data tightly packed into vectors.\\
 & Forward rendering uses this to compute per-vertex lighting.\\
 & 正向渲染中,最多有4个点光源会以逐顶点渲染的方式被计算。\\
\hline
\end{tabular}
\end{center}

\subsubsection{mul(UNITY\_MATRIX\_MVP,v)跟ComputeScreenPos的区别}
\label{sec-1-3-1}
一个是model position->projection position 投影坐标
一个是projection position->screen position\ldots{}屏幕坐标
投影坐标系->屏幕坐标系这是最简单的。2D坐标变换。也不多说。
使用例子:
\begin{minted}[linenos=true]{csharp}
o.position = mul(UNITY_MATRIX_MVP, v.vertex);

o.proj0 = ComputeScreenPos(o.position);

COMPUTE_EYEDEPTH(o.proj0.z);
\end{minted}




\section{Reference}
\label{sec-2}
\subsection{Shader}
\label{sec-2-1}
\begin{itemize}
\item Shader学习:描边Outline初步
\begin{itemize}
\item \url{https://zhuanlan.zhihu.com/p/55337247}
\end{itemize}
\end{itemize}

\subsection{Save Game progress}
\label{sec-2-2}
\begin{itemize}
\item How to Save and Load Your Players' Progress in Unity 2014
\begin{itemize}
\item \url{https://gamedevelopment.tutsplus.com/tutorials/how-to-save-and-load-your-players-progress-in-unity--cms-20934}
\end{itemize}
\item FireBase数据库 保存数据 游戏得分排行榜等
\begin{itemize}
\item \url{https://firebase.google.com/docs/database/unity/save-data?hl=zh-cn}
\end{itemize}
\item 轻量级 Unity3D-小规模初始化数据的存储和读取
\begin{itemize}
\item \url{https://blog.csdn.net/wuyt2008/article/details/60955491}
\end{itemize}
\item 适用于 Unity 的 AWS 移动开发工具包入门  ====》》》不知道这个是做什么用的????
\begin{itemize}
\item \url{https://docs.aws.amazon.com/zh_cn/mobile/sdkforunity/developerguide/getting-started-unity.html}
\end{itemize}
\item 数据存储开发指南 · Unity 2019
\begin{itemize}
\item \url{https://leancloud.cn/docs/unity_guide.html}
\end{itemize}
\end{itemize}
\subsection{Unity LineRender}
\label{sec-2-3}
\begin{itemize}
\item Unity 几种画线方式 GL(Graphics Library) matrix etc
\begin{itemize}
\item \url{https://blog.csdn.net/ldy597321444/article/details/78031284}
\end{itemize}
\item Unity3D点击绘制二维模型线和三维模型线
\begin{itemize}
\item \url{https://blog.csdn.net/zxy13826134783/article/details/80114727}
\end{itemize}
\item Edge Detection via Shader not Image Effect
\begin{itemize}
\item \url{https://forum.unity.com/threads/edge-detection-via-shader-not-image-effect.368922/}
\end{itemize}
\item Image Effect: Edge Detect Normals Colours [rel]
\begin{itemize}
\item \url{https://forum.unity.com/threads/image-effect-edge-detect-normals-colours-rel.310280/?_ga=2.193847467.70482378.1568958437-524766537.1568434661}
\end{itemize}
\end{itemize}

\subsection{FireBase数据库}
\label{sec-2-4}
\begin{itemize}
\item 在 Unity 中开始使用 Firebase 身份验证
\begin{itemize}
\item \url{https://firebase.google.com/docs/auth/unity/start?hl=zh-cn}
\item github: \url{https://github.com/google/mechahamster}
\end{itemize}
\item 将 Firebase 添加到您的 Unity 项目
\begin{itemize}
\item \url{https://firebase.google.com/docs/unity/setup?hl=zh-cn}
\end{itemize}
\item \begin{itemize}
\item 
\end{itemize}
\item \begin{itemize}
\item 
\end{itemize}
\end{itemize}
\subsection{Admob广告}
\label{sec-2-5}
\begin{itemize}
\item Admob + Firebase Get Started in Android Studio
\begin{itemize}
\item \url{https://firebase.google.com/docs/admob/android/quick-start}
\end{itemize}
\item Android Google AdMob 广告接入示例
\begin{itemize}
\item \url{https://github.com/googleads/googleads-mobile-android-examples}
\end{itemize}
\item Android Firebase接入(四)-- AdMob广告
\begin{itemize}
\item \url{https://blog.csdn.net/AlpinistWang/article/details/87438367}
\end{itemize}
\begin{minted}[linenos=true]{csharp}
public class MainActivity extends Activity {
    private InterstitialAd interstitialAd;
    @Override
        protected void onCreate(Bundle savedInstanceState) {
        super.onCreate(savedInstanceState);
        setContentView(R.layout.activity_main);
        showInterstitialAd();
    }
    private void showInterstitialAd(){
        interstitialAd = new InterstitialAd(this);
        interstitialAd.setAdUnitId("你的插屏广告id"));
    interstitialAd.loadAd(new AdRequest.Builder().build());
    interstitialAd.setAdListener(new AdListener(){
            @Override
            public void onAdLoaded() {
                super.onAdLoaded();
                if (interstitialAd.isLoaded()) {
                    interstitialAd.show();
                }
            }
        });
}
\end{minted}
\begin{itemize}
\item 笔者推荐将showInterstitialAd放在BaseActivity中,然后在继承了BaseActivity的页面中直接调用即可展示广告。加载横幅广告和激励视频广告是类似的。
\end{itemize}
\item \begin{itemize}
\item 
\end{itemize}
\item \begin{itemize}
\item 
\end{itemize}
\end{itemize}

\subsection{Edge Detection}
\label{sec-2-6}
\begin{itemize}
\item Outline Shader 有资源,手把手地教 using unity 原始为2018.3.3f1 好好学习一下
\begin{itemize}
\item \url{https://roystan.net/articles/outline-shader.html}
\item UnityOutlineShader-skeleton.zip
\item github: \url{https://github.com/IronWarrior/UnityOutlineShader}
\end{itemize}
\item CommandBuffer与ImageEffect实践-----Outline
\begin{itemize}
\item \url{https://www.wonderm.cc/2019/05/26/CommandBufferAndImageEffect-Outline/}
\end{itemize}
\item CommandBuffer\_01 标记特殊区域
\begin{itemize}
\item \url{https://www.wonderm.cc/2019/03/04/CommandBuffer-01/}
\end{itemize}

\item 关于Unity Shader的一些心得体会\textasciitilde{} GitHub
\begin{itemize}
\item Sjm-Shader-Collection/Volume 09 EdgeDetection详解边缘检测/Script/
\end{itemize}
\item \url{https://github.com/swordjoinmagic/Sjm-Shader-Collection}
\begin{itemize}
\item \url{https://github.com/swordjoinmagic/Sjm-Shader-Collection/blob/master/Volume\%2009\%20EdgeDetection\%E8\%AF\%A6\%E8\%A7\%A3\%E8\%BE\%B9\%E7\%BC\%98\%E6\%A3\%80\%E6\%B5\%8B/Script/BulletTimeStartWithEdgeDetection.cs}
\item 目标物体的边缘检测
\end{itemize}
\item Unity3D 卡通渲染 基于退化四边形的实时描边 - L-灵刃
\begin{itemize}
\item \url{https://www.w3xue.com/exp/article/20199/53598.html}
\item GitHub: \url{https://github.com/L-LingRen/UnitySimpleCartoonLine} 下载
\end{itemize}
\item 【Unity Shaders】法线纹理(Normal Mapping)的实现细节
\begin{itemize}
\item \url{https://blog.csdn.net/candycat1992/article/details/41605257}
\end{itemize}
\item Sobel边缘检测算法
\begin{itemize}
\item \url{https://blog.csdn.net/tianhai110/article/details/5663756}
\end{itemize}
\item unity3d shader之Roberts,Sobel,Canny 三种边缘检测方法
\begin{itemize}
\item \url{http://www.voidcn.com/article/p-mqllafvg-xt.html}
\end{itemize}
\item Unity Shader-边缘检测效果(基于颜色,基于深度法线,边缘流光效果,转场效果)
\begin{itemize}
\item \url{https://gameinstitute.qq.com/community/detail/128772}
\end{itemize}
\item Unity Shader学习笔记(26)边缘检测(深度和法线纹理)
\begin{itemize}
\item \url{https://gameinstitute.qq.com/community/detail/121022}
\end{itemize}
\item 彻底理解数字图像处理中的卷积-以Sobel算子为例
\begin{itemize}
\item \url{https://my.oschina.net/freeblues/blog/727561}
\end{itemize}
\item shader实现屏幕处理效果——边缘检测
\begin{itemize}
\item \url{https://www.jianshu.com/p/fa7cea5f6a72}
\end{itemize}
\item Unity3D开发之边缘检测Sobel算子的一些个人观点
\begin{itemize}
\item \url{https://blog.csdn.net/qq_33994566/article/details/79180058}
\end{itemize}
\item \begin{itemize}
\item 
\end{itemize}
\item \begin{itemize}
\item 
\end{itemize}
\end{itemize}
\subsection{Mesh}
\label{sec-2-7}
\begin{itemize}
\item Runtime Mesh Manipulation With Unity
\begin{itemize}
\item \url{https://www.raywenderlich.com/5128-runtime-mesh-manipulation-with-unity}
\item 
\end{itemize}
\end{itemize}
\subsection{Unity PostProcessing}
\label{sec-2-8}
\begin{itemize}
\item Unity PostProcessing Stack v2源码分析系列
\begin{itemize}
\item \url{https://blog.csdn.net/wolf96/article/details/82796174}
\end{itemize}
\item MMD联动Unity学习笔记 Vol.5.1 Post Processing Stack v2
\begin{itemize}
\item \url{https://www.bilibili.com/read/cv2780283/}
\item 和一个小视频可以参考学习一下
\end{itemize}
\end{itemize}






\section{Reference}
\label{sec-3}
\subsection{Save Game progress}
\label{sec-3-1}
\begin{itemize}
\item How to Save and Load Your Players' Progress in Unity 2014
\begin{itemize}
\item \url{https://gamedevelopment.tutsplus.com/tutorials/how-to-save-and-load-your-players-progress-in-unity--cms-20934}
\end{itemize}
\item FireBase数据库 保存数据 游戏得分排行榜等
\begin{itemize}
\item \url{https://firebase.google.com/docs/database/unity/save-data?hl=zh-cn}
\end{itemize}
\item 轻量级 Unity3D-小规模初始化数据的存储和读取
\begin{itemize}
\item \url{https://blog.csdn.net/wuyt2008/article/details/60955491}
\end{itemize}
\item 适用于 Unity 的 AWS 移动开发工具包入门  ====》》》不知道这个是做什么用的????
\begin{itemize}
\item \url{https://docs.aws.amazon.com/zh_cn/mobile/sdkforunity/developerguide/getting-started-unity.html}
\end{itemize}
\item 数据存储开发指南 · Unity 2019
\begin{itemize}
\item \url{https://leancloud.cn/docs/unity_guide.html}
\end{itemize}
\end{itemize}
\subsection{Unity LineRender}
\label{sec-3-2}
\begin{itemize}
\item Unity 几种画线方式 GL(Graphics Library) matrix etc
\begin{itemize}
\item \url{https://blog.csdn.net/ldy597321444/article/details/78031284}
\end{itemize}
\item Unity3D点击绘制二维模型线和三维模型线
\begin{itemize}
\item \url{https://blog.csdn.net/zxy13826134783/article/details/80114727}
\end{itemize}
\item Edge Detection via Shader not Image Effect
\begin{itemize}
\item \url{https://forum.unity.com/threads/edge-detection-via-shader-not-image-effect.368922/}
\end{itemize}
\item Image Effect: Edge Detect Normals Colours [rel]
\begin{itemize}
\item \url{https://forum.unity.com/threads/image-effect-edge-detect-normals-colours-rel.310280/?_ga=2.193847467.70482378.1568958437-524766537.1568434661}
\end{itemize}
\end{itemize}

\subsection{FireBase数据库}
\label{sec-3-3}
\begin{itemize}
\item 在 Unity 中开始使用 Firebase 身份验证
\begin{itemize}
\item \url{https://firebase.google.com/docs/auth/unity/start?hl=zh-cn}
\item github: \url{https://github.com/google/mechahamster}
\end{itemize}
\item 将 Firebase 添加到您的 Unity 项目
\begin{itemize}
\item \url{https://firebase.google.com/docs/unity/setup?hl=zh-cn}
\end{itemize}
\item \begin{itemize}
\item 
\end{itemize}
\item \begin{itemize}
\item 
\end{itemize}
\end{itemize}
\subsection{Admob广告}
\label{sec-3-4}
\begin{itemize}
\item Android Google AdMob 广告接入示例
\begin{itemize}
\item \url{https://github.com/googleads/googleads-mobile-android-examples}
\end{itemize}
\item Android Firebase接入(四)-- AdMob广告
\begin{itemize}
\item \url{https://blog.csdn.net/AlpinistWang/article/details/87438367}
\end{itemize}
\begin{minted}[linenos=true]{csharp}
public class MainActivity extends Activity {
    private InterstitialAd interstitialAd;
    @Override
        protected void onCreate(Bundle savedInstanceState) {
        super.onCreate(savedInstanceState);
        setContentView(R.layout.activity_main);
        showInterstitialAd();
    }
    private void showInterstitialAd(){
        interstitialAd = new InterstitialAd(this);
        interstitialAd.setAdUnitId("你的插屏广告id"));
    interstitialAd.loadAd(new AdRequest.Builder().build());
    interstitialAd.setAdListener(new AdListener(){
            @Override
            public void onAdLoaded() {
                super.onAdLoaded();
                if (interstitialAd.isLoaded()) {
                    interstitialAd.show();
                }
            }
        });
}
\end{minted}
\begin{itemize}
\item 笔者推荐将showInterstitialAd放在BaseActivity中,然后在继承了BaseActivity的页面中直接调用即可展示广告。加载横幅广告和激励视频广告是类似的。
\end{itemize}
\item \begin{itemize}
\item 
\end{itemize}
\item \begin{itemize}
\item 
\end{itemize}
\end{itemize}



\subsection{Edge Detection}
\label{sec-3-5}
\begin{itemize}
\item 关于Unity Shader的一些心得体会\textasciitilde{} GitHub
\begin{itemize}
\item Sjm-Shader-Collection/Volume 09 EdgeDetection详解边缘检测/Script/
\end{itemize}
\item \url{https://github.com/swordjoinmagic/Sjm-Shader-Collection}
\begin{itemize}
\item \url{https://github.com/swordjoinmagic/Sjm-Shader-Collection/blob/master/Volume\%2009\%20EdgeDetection\%E8\%AF\%A6\%E8\%A7\%A3\%E8\%BE\%B9\%E7\%BC\%98\%E6\%A3\%80\%E6\%B5\%8B/Script/BulletTimeStartWithEdgeDetection.cs}
\item 目标物体的边缘检测
\end{itemize}
\item Unity3D 卡通渲染 基于退化四边形的实时描边 - L-灵刃
\begin{itemize}
\item \url{https://www.w3xue.com/exp/article/20199/53598.html}
\item GitHub: \url{https://github.com/L-LingRen/UnitySimpleCartoonLine} 下载
\end{itemize}
\item 【Unity Shaders】法线纹理(Normal Mapping)的实现细节
\begin{itemize}
\item \url{https://blog.csdn.net/candycat1992/article/details/41605257}
\end{itemize}
\item Sobel边缘检测算法
\begin{itemize}
\item \url{https://blog.csdn.net/tianhai110/article/details/5663756}
\end{itemize}
\item unity3d shader之Roberts,Sobel,Canny 三种边缘检测方法
\begin{itemize}
\item \url{http://www.voidcn.com/article/p-mqllafvg-xt.html}
\end{itemize}
\item Unity Shader-边缘检测效果(基于颜色,基于深度法线,边缘流光效果,转场效果)
\begin{itemize}
\item \url{https://gameinstitute.qq.com/community/detail/128772}
\end{itemize}
\item Unity Shader学习笔记(26)边缘检测(深度和法线纹理)
\begin{itemize}
\item \url{https://gameinstitute.qq.com/community/detail/121022}
\end{itemize}
\item 彻底理解数字图像处理中的卷积-以Sobel算子为例
\begin{itemize}
\item \url{https://my.oschina.net/freeblues/blog/727561}
\end{itemize}
\item shader实现屏幕处理效果——边缘检测
\begin{itemize}
\item \url{https://www.jianshu.com/p/fa7cea5f6a72}
\end{itemize}
\item Unity3D开发之边缘检测Sobel算子的一些个人观点
\begin{itemize}
\item \url{https://blog.csdn.net/qq_33994566/article/details/79180058}
\end{itemize}
\item \begin{itemize}
\item 
\end{itemize}
\item \begin{itemize}
\item 
\end{itemize}
\item \begin{itemize}
\item 
\end{itemize}
\item \begin{itemize}
\item 
\end{itemize}
\item \begin{itemize}
\item 
\end{itemize}
\end{itemize}
% Emacs 25.3.1 (Org mode 8.2.7c)
\end{document}