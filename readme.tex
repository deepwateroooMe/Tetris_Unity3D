% Created 2022-09-27 Tue 19:41
\documentclass[9pt, b5paper]{article}
\usepackage{xeCJK}
\usepackage{minted}
\usepackage[T1]{fontenc}
\usepackage[scaled]{beraserif}
\usepackage[scaled]{berasans}
\usepackage[scaled]{beramono}
\usepackage{graphicx}
\usepackage{xcolor}
\usepackage{multirow}
\usepackage{multicol}
\usepackage{float}
\usepackage{textcomp}
\usepackage{algorithm}
\usepackage{algorithmic}
\usepackage{latexsym}
\usepackage{natbib}
\usepackage{geometry}
\geometry{left=1.2cm,right=1.2cm,top=1.5cm,bottom=1.2cm}
\newminted{common-lisp}{fontsize=\footnotesize} 
\usepackage[xetex,colorlinks=true,CJKbookmarks=true,linkcolor=blue,urlcolor=blue,menucolor=blue]{hyperref}
\author{deepwaterooo}
\date{\today}
\title{Unity Android SDK/NDK 俄罗斯方块砖3D小游戏}
\hypersetup{
  pdfkeywords={},
  pdfsubject={},
  pdfcreator={Emacs 27.1 (Org mode 8.2.7c)}}
\begin{document}

\maketitle
\setcounter{tocdepth}{5}
\tableofcontents


\section{模块搭建}
\label{sec-1}
\begin{itemize}
\item \# only inlcude two levels in TOC
\item ILRuntime的消化理解,以及与MVVM同用时的搭配理解消化
\item 热更新模块服务器模块的理解与消化搭建:
\end{itemize}

\section{把原理弄懂}
\label{sec-2}
\begin{itemize}
\item 热更新模块的实充:以前的设计模式和实现的功能还是比较完整的;现在更成熟一点儿,需要把热更新模块补充出来;
\item ILRuntime + MVVM框架设计:两者结合,前几年的时候没能把MVVM理解透彻;
\item 上次前几年主要的难点:好像是在把MVVM双向数据绑定理解得不透彻;那么这次应该就狠没有问题了,更该寻求更好的设计与解决方案
\item 性能优化:另外是对其实高级开发的越来越熟悉,希望应用的性能表现,尤其是渲染性能与速度等、这些更为高级和深入的特性成为这次二次开发的重点。

\item 现在是把自己几年前的写的游戏全忘记了,需要回去把自己的源码找出来,再读一读熟悉一下自己的源码,了解当时设计的估缺点,由此改进更将
\end{itemize}

\section{环境弄得比较好的包括:}
\label{sec-3}
\begin{itemize}
\item 输入法的搭建:终于用到了自己之前用过的好用的输入法
\item 这两天开车疲累,最迟明天中午会去南湾找房间出租,尽快解决搬家的问题;昨天晚上回来得太晚了,一路辛苦,路上只差睡着,回到家里补觉补了好多个小时。
\item 小电脑,笔记本电脑里的游戏环境搭建,今天下午去图书馆里弄(今天下午去图书馆里把需要借助快速网络来完成的事情都搭建好;家里被恶房东故意整了个腾腾慢的网,故意阻碍别人的发展,谁还愿意再这样的环境中继续住下去呢?!!!)
\end{itemize}

\section{ILRuntime 库的系统再深入理解}
\label{sec-4}
\subsection{ILRuntime基本原理}
\label{sec-4-1}
\begin{itemize}
\item ILRuntime借助Mono.Cecil库来读取DLL的PE信息,以及当中类型的所有信息,最终得到方法的IL汇编码,然后通过内置的IL解译执行虚拟机来执行DLL中的代码。IL解释器代码在ILIntepreter.cs,通过Opcode来逐语句执行机器码,解释器的代码有四千多行。
\end{itemize}

\includegraphics[width=.9\linewidth]{./pic/readme_20220926_094936.png}
\subsection{ILRuntime热更流程}
\label{sec-4-2}

\includegraphics[width=.9\linewidth]{./pic/readme_20220926_095022.png}
\subsection{ILRuntime主要限制}
\label{sec-4-3}

\includegraphics[width=.9\linewidth]{./pic/readme_20220926_095555.png}
\subsection{ILRuntime启动调试}
\label{sec-4-4}
\begin{itemize}
\item ILRuntime建议全局只创建一个AppDomain,在函数入口添加代码启动调试服务
\end{itemize}
\begin{minted}[fontsize=\scriptsize,linenos=false]{csharp}
appdomain.DebugService.StartDebugService(56000)
\end{minted}
\begin{itemize}
\item 运行主工程(Unity工程)
\item 在热更的VS工程中 点击 - 调试 - 附加到ILRuntime调试,注意使用一样的端口
\item 如果使用VS2015的话需要Visual Studio 2015 Update3以上版本
\end{itemize}
\subsection{线上项目和资料}
\label{sec-4-5}
\begin{itemize}
\item 掌趣很多项目都是使用ILRuntime开发,并上线运营,比如:真红之刃,境·界 灵压对决,全民奇迹2,龙族世界,热血足球
\item 初音未来:梦幻歌姬 使用补丁方式:\url{https://github.com/wuxiongbin/XIL}
\item 本文流程图摘自:ILRuntime的QQ群的《ILRuntime热更框架.docx》(by a 704757217)
\item Unity实现c\#热更新方案探究(三): \url{https://zhuanlan.zhihu.com/p/37375372}
\end{itemize}

\section{Framework.Core核心理解 ?}
\label{sec-5}
\subsection{Adaptor}
\label{sec-5-1}
\subsubsection{}
\label{sec-5-1-1}
\subsubsection{}
\label{sec-5-1-2}
\subsubsection{}
\label{sec-5-1-3}
\subsection{Factory}
\label{sec-5-2}
\subsection{Inject}
\label{sec-5-3}
\subsection{Wrap}
\label{sec-5-4}
\subsection{GameApplication: 游戏入口类}
\label{sec-5-5}
\begin{minted}[fontsize=\scriptsize,linenos=false]{csharp}
// 游戏入口类
public class GameApplication : MonoBehaviour {
    private const string TAG = "GameApplication";
        
    private static GameApplication _instance;
    public static GameApplication Instance {
        get {
            return _instance;
        }
    }
    public IHotFixMain HotFix {
        get;
        set;
    }

// 五个用户、客户端可配置变量,以及它们的根据用户配置(读取系统中用户配置文件里的相关五个配置)
    // 是否使用PDB调试信息
    public bool usePDB = false;
    // 是否使用ILRuntime模式热更新
    public bool useILRuntime = false;
    // 是否使用本地资源
    public bool useLocal = false;
    // 资源服务器路径
    public string webRoot = string.Empty;
    // 强制登录
    public bool forceLogin = false;
    
    public ScreenRaycaster ScreenRaycaster {
        get;
        private set;
    }
// 这个公用方法应该是跟游戏中时常需要接入一个或是多个游戏SDK相关,提供便利接入方法
    public ShareSDK ShareSDK { 
        get;
        private set;
    }
    void Awake() {
        _instance = this;
        ScreenRaycaster = GameObject.Find("Gestures").GetComponent<ScreenRaycaster>();
        DontDestroyOnLoad(gameObject); // <<<<<<<<<<<<<<<<<<<< 我自己的游戏中实现过这个,可是现在回想得好辛苦呀。。。爱表哥爱生活爱游戏
        //InitializeClientConfig();
        //InitializeSDKs();
        CoroutineHelper.StartCoroutine(Initialize()); // 协程:
#region TestSamples
        //FingerEventTemp.Instance.RegisterGestureEvents();
        //TestNTS.Instance.TestLinesAngle();
        //GeometryManager.Instance.Test();
#endregion
    }
// 客户端的配置是写在一个文件里的,需要的话直接读就可以了,安卓系统很多地方也这样
    void InitializeClientConfig() {
        var str = FileHelp.ReadString("ClientConfig.txt"); 
        if (!string.IsNullOrEmpty(str)) {
            JsonObject jsonObject = JsonSerializer.Deserialize(str) as JsonObject;
            if (jsonObject != null) {
                if (jsonObject.ContainsKey("usePDB")) 
                    usePDB = (bool)jsonObject["usePDB"];
                if (jsonObject.ContainsKey("useILRuntime")) 
                    usePDB = (bool)jsonObject["useILRuntime"];
                if (jsonObject.ContainsKey("useLocal")) 
                    usePDB = (bool)jsonObject["useLocal"];
                if (jsonObject.ContainsKey("webRoot")) 
                    ResourceConstant.ResourceWebRoot = jsonObject["webRoot"].ToString();
                if (jsonObject.ContainsKey("forceLogin")) 
                    forceLogin = (bool)jsonObject["forceLogin"];
            }
        }
    }

    void InitializeSDKs() {
        if (Application.platform == RuntimePlatform.IPhonePlayer || Application.platform == RuntimePlatform.Android) 
            InitializeShareSDK();
    }
    void InitializeShareSDK() {
        ShareSDK = GetComponent<ShareSDK>();
        ShareSDK.authHandler = AuthResultHandler;
        ShareSDK.Authorize(PlatformType.WeChat);
    }
    // ShareSDK执行授权回调
    void AuthResultHandler(int reqID, ResponseState state, PlatformType type, Hashtable result) {
        if (state == ResponseState.Success) {
            Debug.Log("ShareSDK authorize success!");
        } else if (state == ResponseState.Fail) {
            Debug.Log("fail! error code = " + result["error_code"] + "; error msg = " + result["error_msg"]);
        } else if (state == ResponseState.Cancel) {
            Debug.Log("cancel!");
        }
    }

// 协程是说:游戏启动时,给这个控件(gameObject)加载运行时脚本ResourceMap; 加载完毕自动触发StartHotFix()函数热更新回调
    IEnumerator Initialize() { 
        ResourceMap resourceMap = gameObject.AddComponent<ResourceMap>();
        resourceMap.OnInitializeSuccess += StartHotFix;
        ResourceConstant.Loader = resourceMap;
        yield return new WaitForEndOfFrame();
    }
    public void StartHotFix() {
        Debug.Log(TAG + ": StartHotFix()");
        Debug.Log(TAG + " useILRuntime: " + useILRuntime); 
        if (Application.platform == RuntimePlatform.IPhonePlayer) {
            HotFix = HotFixILRunTime.Instance;
        } else {
            if (useILRuntime) { // 使用热更新程序集的实例
                HotFix = HotFixILRunTime.Instance;
            } else { // 这里是,还需要再理解消化一下
                HotFix = HotFixReflector.Instance;
            }
        }
    }
}
\end{minted}
\subsection{}
\label{sec-5-6}
\subsection{HotFixILRunTime : SingletonMono<HotFixILRunTime>, IHotFixMain: 我忘记了什么叫SingletonMono ?}
\label{sec-5-7}
\begin{minted}[fontsize=\scriptsize,linenos=false]{csharp}
public class HotFixILRunTime : SingletonMono<HotFixILRunTime>, IHotFixMain { // 我忘记了什么叫SingletonMono ?
    public static ILRuntime.Runtime.Enviorment.AppDomain appDomain;

    void Start() {
        appDomain = new ILRuntime.Runtime.Enviorment.AppDomain();
#if UNITY_EDITOR
        appDomain.UnityMainThreadID = System.Threading.Thread.CurrentThread.ManagedThreadId;
#endif
        // 调用资源管理器加载这两个程序集:HotFix.dll + HotFix.pdb
        TextAsset dllAsset = ResourceConstant.Loader.LoadAsset<TextAsset>("HotFix.dll", "HotFix.dll"); // 同步加载
        var msDll = new System.IO.MemoryStream(dllAsset.bytes); // 这里涉及到内存管理吗?以前我不曾涉及,现在长大了,应该把它理解消化好
        if (GameApplication.Instance.usePDB) {
            ResourceConstant.Loader.LoadAssetAsyn<TextAsset>("HotFix.pdb", "HotFix.pdb", (pdbAsset) => { // 异步加载
                var msPdb = new System.IO.MemoryStream(pdbAsset.bytes);
                appDomain.LoadAssembly(msDll, msPdb, new Mono.Cecil.Mdb.MdbReaderProvider());
                StartApplication(); // <<<<<<<<<<<<<<<<<<<< 
            }, EAssetBundleUnloadLevel.ChangeSceneOver);
        } else {
            appDomain.LoadAssembly(msDll, null, new Mono.Cecil.Mdb.MdbReaderProvider());
            StartApplication();
        }
    }
    void StartApplication() {
        InitializeILRunTimeHotFixSetting();
        DoStaticMethod("HotFix.HotFixMain", "Start");
    }
    void InitializeILRunTimeHotFixSetting() {
        InitializeDelegateSetting(); // 方法被我搬到了文件的最后,太长比较简单
        InitializeCLRBindSetting();
        InitializeAdapterSetting();
        InitializeValueTypeSetting();
    }
    unsafe void InitializeCLRBindSetting() {
        foreach (var i in typeof(System.Activator).GetMethods()) {
            // 找到名字为CreateInstance,并且是泛型方法的方法定义 ?
            if (i.Name == "CreateInstance" && i.IsGenericMethodDefinition) 
                appDomain.RegisterCLRMethodRedirection(i, CreateInstance); // 方法重定向?再理解消化一下,不太懂还
        }
    }

    void InitializeAdapterSetting() {
        appDomain.RegisterCrossBindingAdaptor(new ViewModelBaseAdapter());  // 这个看过了
        appDomain.RegisterCrossBindingAdaptor(new UnityGuiViewAdapter());  // 这个看过了
        appDomain.RegisterCrossBindingAdaptor(new ModuleBaseAdapter());   // 这个看过了
// 下面的这几适配器还不知道是怎么回事呀
        appDomain.RegisterCrossBindingAdaptor(new IEnumeratorObjectAdaptor());
        appDomain.RegisterCrossBindingAdaptor(new MonoBehaviourAdapter());
        appDomain.RegisterCrossBindingAdaptor(new InterfaceCrossBindingAdaptor());
    }
    void InitializeValueTypeSetting() {
        appDomain.RegisterValueTypeBinder(typeof(Vector3), new Vector3Binder());
        appDomain.RegisterValueTypeBinder(typeof(Vector2), new Vector2Binder());
        appDomain.RegisterValueTypeBinder(typeof(Quaternion), new QuaternionBinder());
    }

// 定义如何调用热更新程序集里的静态入口方法
    object DoStaticMethod(string type, string method) { 
        var hotfixType = appDomain.GetType(type);
        var staticMethod = hotfixType.GetMethod(method, 0);
        return appDomain.Invoke(staticMethod, null, null);
    }

// 定义热更新的两个方法的特用实现    
#region Override
    public Type LoadType(string typeName) {
        if (appDomain.LoadedTypes.ContainsKey(typeName)) {
            return appDomain.LoadedTypes[typeName].ReflectionType;
        }
        return null;
    }
    public object CreateInstance(string typeName) {
        ILType type = (ILType)appDomain.LoadedTypes[typeName];
        var instance = type.Instantiate();
        return instance;
    }
#endregion

    // 公用静态方法的实现
    public unsafe static StackObject* CreateInstance(ILIntepreter intp, StackObject* esp, IList<object> mStack, CLRMethod method, bool isNewObj) {
        // 获取泛型参数<T>的实际类型
        IType[] genericArguments = method.GenericArguments;
        if (genericArguments != null && genericArguments.Length == 1) {
            var t = genericArguments[0];
            if (t is ILType)// 如果T是热更DLL里的类型 {
                // 通过ILRuntime的接口来创建实例
                return ILIntepreter.PushObject(esp, mStack, ((ILType)t).Instantiate());
        } else
            return ILIntepreter.PushObject(esp, mStack, Activator.CreateInstance(t.TypeForCLR));// 通过系统反射接口创建实例
    } else
          throw new EntryPointNotFoundException();
}

// 不难猜测理解:所有需要热更新的方法类型(以不同的参数类型以及返回类型作区分),都要在这里向程序集的代理管理器注册相关方法回调代理    
    void InitializeDelegateSetting() {
        appDomain.DelegateManager.RegisterMethodDelegate<int>();
        appDomain.DelegateManager.RegisterFunctionDelegate<int, string>();
        appDomain.DelegateManager.RegisterMethodDelegate<string>();
        appDomain.DelegateManager.RegisterMethodDelegate<int, int>();
        appDomain.DelegateManager.RegisterMethodDelegate<List<int>, List<int>>();
        appDomain.DelegateManager.RegisterMethodDelegate<string, string>();
        appDomain.DelegateManager.RegisterMethodDelegate<object, MessageArgs<object>>();
        appDomain.DelegateManager.RegisterMethodDelegate<object, MessageArgs<ILTypeInstance>>();
        appDomain.DelegateManager.RegisterMethodDelegate<GameObject>();
        appDomain.DelegateManager.RegisterMethodDelegate<UnityEngine.Networking.UnityWebRequest>();
        appDomain.DelegateManager.RegisterMethodDelegate<TMP_FontAsset>();
        appDomain.DelegateManager.RegisterMethodDelegate<Font>();
        appDomain.DelegateManager.RegisterMethodDelegate<AnimationClip>();
        appDomain.DelegateManager.RegisterMethodDelegate<AnimatorOverrideController>();
        appDomain.DelegateManager.RegisterMethodDelegate<RuntimeAnimatorController>();
        appDomain.DelegateManager.RegisterMethodDelegate<AudioClip>();
        appDomain.DelegateManager.RegisterMethodDelegate<Material>();
        appDomain.DelegateManager.RegisterMethodDelegate<TextAsset>();
        appDomain.DelegateManager.RegisterMethodDelegate<Sprite>();
        appDomain.DelegateManager.RegisterMethodDelegate<Texture2D>();
        appDomain.DelegateManager.RegisterMethodDelegate<TapGesture>();
        appDomain.DelegateManager.RegisterMethodDelegate<LongPressGesture>();
        appDomain.DelegateManager.RegisterMethodDelegate<DragGesture>();
        appDomain.DelegateManager.RegisterMethodDelegate<PinchGesture>();
        appDomain.DelegateManager.RegisterMethodDelegate<Exception>();
        appDomain.DelegateManager.RegisterFunctionDelegate<GameObject, GameObject>();
        appDomain.DelegateManager.RegisterFunctionDelegate<ILTypeInstance, ILTypeInstance, int>();
       // 在Unity的程序集与热更新的程序集跨域之间,这里给出了方法代理转换的实现逻辑,几座小桥流水人家,爱表哥爱生活
        appDomain.DelegateManager.RegisterDelegateConvertor<UnityAction>((action) => {
            return new UnityAction(() => {
                ((Action)action)();
            });
        });
        appDomain.DelegateManager.RegisterDelegateConvertor<UnityAction<bool>>((action) => {
            return new UnityAction<bool>((b) => {
                ((Action<bool>)action)(b);
            });
        });
        appDomain.DelegateManager.RegisterDelegateConvertor<UnityAction<int>>((action) => {
            return new UnityAction<int>((b) => {
                ((Action<int>)action)(b);
            });
        });
        appDomain.DelegateManager.RegisterDelegateConvertor<UnityAction<long>>((action) => {
            return new UnityAction<long>((b) => {
                ((Action<long>)action)(b);
            });
        });
        appDomain.DelegateManager.RegisterDelegateConvertor<UnityAction<float>>((action) => {
            return new UnityAction<float>((b) => {
                ((Action<float>)action)(b);
            });
        });
        appDomain.DelegateManager.RegisterDelegateConvertor<UnityAction<string>>((action) => {
            return new UnityAction<string>((b) => {
                ((Action<string>)action)(b);
            });
        });
        appDomain.DelegateManager.RegisterDelegateConvertor<UnityAction<BaseEventData>>((action) => {
            return new UnityAction<BaseEventData>((b) => {
                ((Action<BaseEventData>)action)(b);
            });
        });
        appDomain.DelegateManager.RegisterDelegateConvertor<GestureRecognizerTS<TapGesture>.GestureEventHandler>((action) => {
            return new GestureRecognizerTS<TapGesture>.GestureEventHandler((gesture) => {
                ((Action<TapGesture>)action)(gesture);
            });
        });
        appDomain.DelegateManager.RegisterDelegateConvertor<GestureRecognizerTS<LongPressGesture>.GestureEventHandler>((action) => {
            return new GestureRecognizerTS<LongPressGesture>.GestureEventHandler((gesture) => {
                ((Action<LongPressGesture>)action)(gesture);
            });
        });
        appDomain.DelegateManager.RegisterDelegateConvertor<GestureRecognizerTS<DragGesture>.GestureEventHandler>((action) => {
            return new GestureRecognizerTS<DragGesture>.GestureEventHandler((gesture) => {
                ((Action<DragGesture>)action)(gesture);
            });
        });
        appDomain.DelegateManager.RegisterDelegateConvertor<GestureRecognizerTS<PinchGesture>.GestureEventHandler>((action) => {
            return new GestureRecognizerTS<PinchGesture>.GestureEventHandler((gesture) => {
                ((Action<PinchGesture>)action)(gesture);
            });
        });
        #if UNITY_IPHONE
                appDomain.DelegateManager.RegisterDelegateConvertor<com.mob.FinishedRecordEvent>((action) => {
            return new com.mob.FinishedRecordEvent((ex) => {
                ((Action<Exception>)action)(ex);
            });
        });
        #endif
        appDomain.DelegateManager.RegisterDelegateConvertor<Comparison<ILTypeInstance>>((action) => {
            return new Comparison<ILTypeInstance>((x, y) => {
                return ((Func<ILTypeInstance, ILTypeInstance, System.Int32>)action)(x, y);
            });
        });
    }
}
\end{minted}
\subsection{IHotFixMain interface}
\label{sec-5-8}
\begin{minted}[fontsize=\scriptsize,linenos=false]{csharp}
public interface IHotFixMain {
    Type LoadType(string typeName);
    object CreateInstance(string typeName);
}
\end{minted}

\section{Framework.MVVM: Unity中定义好的MVVM架构;使用ILRuntime定义实现了必要的跨域跨程序集适配,以及数据观察回调等}
\label{sec-6}
\subsection{DataBinding: MVVM具备双向数据绑定功能;这里这个框架里通过代理观察者模式来实现数据的改变通知与回调}
\label{sec-6-1}
\subsubsection{BindableProperty<T> : 泛型基类,可绑定的属性}
\label{sec-6-1-1}
\begin{minted}[fontsize=\scriptsize,linenos=false]{csharp}
public class BindableProperty<T> { // 泛型基类
    private T _value;

    public Action<T, T> OnValueChanged; // 代理模式,观察者模式?
    public T Value {
        get {
            return _value;
        }
        set {
            if (!Equals(_value, value)) {
                T old = _value;
                _value = value;
                ValueChanged(old, _value);
            }
        }
    }
    void ValueChanged(T oldValue, T newValue) {
        if (OnValueChanged != null) 
            OnValueChanged(oldValue, newValue);
    }

    public override string ToString() {
        return (Value != null ? Value.ToString() : "null");
    }
}
\end{minted}
\subsubsection{ObservableList<T> : IList<T> :可被观察的链表,同样包装成泛型基类}
\label{sec-6-1-2}
\begin{minted}[fontsize=\scriptsize,linenos=false]{csharp}
public class ObservableList<T> : IList<T> { // 可被观察的链表:同样包装成泛型基类

    public Action<List<T>, List<T>> OnValueChanged;
    public Action<T> OnAdd;
    public Action<int, T> OnInsert;
    public Action<T> OnRemove;
    private List<T> _value = new List<T>(); // <<<<<<<<<< 

    public List<T> Value {
        get {
            return _value;
        }
        set {
            if (!Equals(_value, value)) {
                var old = _value;
                _value = value;
                ValueChanged(old, _value);
            }
        }
    }
    void ValueChanged(List<T> oldValue, List<T> newValue) {
        if (OnValueChanged != null) 
            OnValueChanged(oldValue, newValue);
    }

    public IEnumerator<T> GetEnumerator() {
        return _value.GetEnumerator();
    }
    IEnumerator IEnumerable.GetEnumerator() {
        return GetEnumerator();
    }

// 覆写和定义链表的相关必要方法
    public void Add(T item) {
        _value.Add(item);
        if (OnAdd != null) 
            OnAdd(item);
    }
    public void Clear() {
        _value.Clear();
    }
    public bool Contains(T item) {
        return _value.Contains(item);
    }
    public void CopyTo(T[] array, int arrayIndex) {
        _value.CopyTo(array, arrayIndex);
    }
    public bool Remove(T item) {
        if (_value.Remove(item)) {
            if (OnRemove != null) 
                OnRemove(item);
            return true;
        }
        return false;
    }
    public int Count {
        get {
            return _value.Count;
        }
    }
    public bool IsReadOnly {
        get;
        private set;
    }
    public int IndexOf(T item) {
        return _value.IndexOf(item);
    }
    public void Insert(int index, T item) {
        _value.Insert(index, item);
        if (OnInsert != null) {
            OnInsert(index, item);
        }
    }
    public void RemoveAt(int index) {
        _value.RemoveAt(index);
    }
    public T this[int index] {
        get {
            return _value[index];
        }
        set {
            _value[index] = value;
        }
    }
}
\end{minted}
\subsubsection{PropertyBinder<ViewModelBase>: ViewModel(的基类以及继承类)的绑定辅助相关方法定义类}
\label{sec-6-1-3}
\begin{minted}[fontsize=\scriptsize,linenos=false]{csharp}
// ViewModel(的基类以及继承类)的绑定辅助相关方法定义类
public class PropertyBinder<ViewModelBase> { 

    private delegate void BindHandler(ViewModelBase viewModel); // 这里才真正意义上的代理模式吧
    private delegate void UnBindHandler(ViewModelBase viewModel);

    private readonly List<BindHandler> binders = new List<BindHandler>();
    private readonly List<UnBindHandler> unBinders = new List<UnBindHandler>();

    public void Add<TProperty>(string name, string realTypeName, Action<TProperty, TProperty> valueChangedHandler) {
        var fieldInfo = GameApplication.Instance.HotFix.LoadType(realTypeName).GetField(name, BindingFlags.Instance | BindingFlags.Public);
        if (fieldInfo == null) 
            throw new Exception(string.Format("Unable to find bindableproperty field '{0}.{1}'", realTypeName, name));
        binders.Add(viewModel => {
            GetPropertyValue<TProperty>(name, viewModel, realTypeName, fieldInfo).OnValueChanged += valueChangedHandler;
        });
        unBinders.Add(viewModel => {
            GetPropertyValue<TProperty>(name, viewModel, realTypeName, fieldInfo).OnValueChanged -= valueChangedHandler;
        });
    }
    private BindableProperty<TProperty> GetPropertyValue<TProperty>(string name, ViewModelBase viewModel,
                                                                    string realTypeName, FieldInfo fieldInfo) {
        var value = fieldInfo.GetValue(viewModel);
        BindableProperty<TProperty> bindableProperty = value as BindableProperty<TProperty>;
        if (bindableProperty == null) 
            throw new Exception(string.Format("Illegal bindableproperty field '{0}.{1}' ", realTypeName, name));
        return bindableProperty;
    }

// 这里告诉一个常识说:任何一个ViewModel,不管是蕨类还是继承后的子类,它们都有可能有好几个视图与其绑定
    public void Bind(ViewModelBase viewModel) {
        if (viewModel != null) 
            for (int i = 0; i < binders.Count; i++) 
                binders[i](viewModel);
    }
// 这里告诉一个常识说:任何一个ViewModel,不管是蕨类还是继承后的子类,它们都有可能有好几个视图,需要与其解绑定
    public void UnBind(ViewModelBase viewModel) {
        if (viewModel != null) 
            for (int i = 0; i < unBinders.Count; i++) 
                unBinders[i](viewModel);
    }
}
\end{minted}
\subsection{Message: 自定义的消息机制吗?这里还没能联系上下文,完全不知道这里是在做什么?}
\label{sec-6-2}
\subsubsection{MessageArgs<T>: 跨域跨程序集的消息参数泛型类 ?}
\label{sec-6-2-1}
\begin{minted}[fontsize=\scriptsize,linenos=false]{csharp}
public class MessageArgs<T> {
    public T Item {
        get;
        private set;
    }
    public MessageArgs(T item) {
        Item = item;
    }
}
\end{minted}
\subsubsection{MessageAggregator<T> : Singleton<MessageAggregator<T>>: 单例模式的泛型管理类?用一个字典来管理消息?}
\label{sec-6-2-2}
\begin{minted}[fontsize=\scriptsize,linenos=false]{csharp}
// 单例模式的泛型管理类?用一个字典来管理消息(可观察的属性、链表、ViewModel等, 以及它们各自对应的回调) ? 这里要再看不遍,还没有真正看懂
public class MessageAggregator<T> : Singleton<MessageAggregator<T>> {

    private readonly Dictionary<string, Action<object, MessageArgs<T>>> messages =
        new Dictionary<string, Action<object, MessageArgs<T>>>();

    public void Subscribe(string name, Action<object, MessageArgs<T>> handler) {
        if (!messages.ContainsKey(name)) 
            messages.Add(name, handler);
        else 
            messages[name] += handler;
    }
    public void Publish(string name, object sender, MessageArgs<T> args) {
        if (messages.ContainsKey(name) && messages[name] != null) 
            messages[name](sender, args);
    }
}
\end{minted}
\subsection{Module: 这里是模块级别的定义;在模块层面级别上;在ViewModel层面级别上等不面层面上的跨域适配定义与实现}
\label{sec-6-3}
\begin{itemize}
\item 不能像最开始一样把这里误当作Model;这是一个framework,数据是定义在各个不同的具体应用里
\end{itemize}
\subsubsection{ModuleBase.cs}
\label{sec-6-3-1}
\begin{minted}[fontsize=\scriptsize,linenos=false]{csharp}
public abstract class ModuleBase {
    public abstract void OnInitialize();
    public abstract void Excute();
}
\end{minted}
\subsubsection{ModuleBaseAdapter : CrossBindingAdaptor:在模块层面上自定义实现了:模块层面级别的跨域跨程序集适配器}
\label{sec-6-3-2}
\begin{minted}[fontsize=\scriptsize,linenos=false]{csharp}
public class ModuleBaseAdapter : CrossBindingAdaptor { // 继承了CrossBindingAdaptor抽象基类

// 抽象蕨类里的三个抽象方法:需要实现 
    public override Type BaseCLRType { // 继承了CrossBindingAdaptor 抽象基类,就应该需要覆写里面定义过的相关方法,改造成自己需要的样子
        get {
            return typeof(ModuleBase);
        }
    }
    public override Type AdaptorType {
        get {
            return typeof(ModuleBaseAdaptor);
        }
    }
    public override object CreateCLRInstance(ILRuntime.Runtime.Enviorment.AppDomain appdomain, ILTypeInstance instance) {
        return new ModuleBaseAdaptor(appdomain, instance);
    }

    // ModuleBase: Framework.MVVM里定义的基类;CrossBindingAdaptorType是ILRuntime.Runtime.Environment里定义的公共接口类interface
    class ModuleBaseAdaptor : ModuleBase, CrossBindingAdaptorType { // 好久没有写cs代码了,这里看得昏昏乎乎,类里定义类,还是相同的类名,有点儿昏呀?

        ILTypeInstance instance;
        ILRuntime.Runtime.Enviorment.AppDomain appdomain;

// 实现对 ModuleBase里的两个抽象方法的调控掌握
        IMethod _onInitialize;
        bool _onInitializeGot;
        IMethod _excute;
        bool _excuteGot;
        
        public ModuleBaseAdaptor() { }
        public ModuleBaseAdaptor(ILRuntime.Runtime.Enviorment.AppDomain appdomain, ILTypeInstance instance) {
            this.appdomain = appdomain; 
            this.instance = instance;
        }
        
        public ILTypeInstance ILInstance { get { return instance; } } // 返回类型的实体类

// 覆写ModuleBase里的两个抽象方法
        public override void OnInitialize() {
            if (!_onInitializeGot) {
                _onInitialize = instance.Type.GetMethod("OnInitialize");
                _onInitializeGot = true;
            }
            if (_onInitialize != null) {
                appdomain.Invoke(_onInitialize, instance, null);
            }
        }
        public override void Excute() {
            if (!_excuteGot) {
                _excute = instance.Type.GetMethod("Excute");
                _excuteGot = true;
            }
            if (_excute != null) {
                appdomain.Invoke(_excute, instance, null);
            }
        }
    }
}
\end{minted}
\subsection{View}
\label{sec-6-4}
\subsubsection{IView<ViewModelBase>}
\label{sec-6-4-1}
\begin{minted}[fontsize=\scriptsize,linenos=false]{csharp}
public interface IView<ViewModelBase> {
    ViewModelBase BindingContext {
        get;
        set;
    }
    void Reveal(bool immediate = false, Action action = null);
    void Hide(bool immediate = false, Action action = null);
}
\end{minted}
\subsubsection{UnityGuiView: IView<ViewModelBase>, 实现或是覆写基类以及泛型里的各种定义过的方法}
\label{sec-6-4-2}
\begin{minted}[fontsize=\scriptsize,linenos=false]{csharp}
// 继承自抽象基类:便需要实现公用接口类里面所定义的三个接口方法
// 泛型类型是ViewModelBase,便可以实现或是覆写里面定义的各种公用、抽象或是protected方法
public abstract class UnityGuiView : IView<ViewModelBase> { // 仍然是抽象基类:这个类比较重要,明天早上再看一下

private bool _isInitialized; // ViewModelBase 里同共有的

// 自已再定义的两个方法,供实现
    public virtual bool DestoryOnHide {
        get {
            return false;
        }
    }
    public virtual bool IsRoot {
        get {
            return false;
        }
    }

    public static Action SetDownRootIndex;
    public Action CloseOtherRootView;
    protected readonly PropertyBinder<ViewModelBase> binder = new PropertyBinder<ViewModelBase>();
    public readonly BindableProperty<ViewModelBase> viewModelProperty = new BindableProperty<ViewModelBase>();

// 实现了抽象接口类,便需要实现里面的所有定义过的接口方法:公用接口类里定义了这三个方法
    public Action RevealedAction {
        get;
        set;
    }
    public Action HiddenAction {
        get;
        set;
    }
    public ViewModelBase BindingContext { // 实现了抽象接口类,便需要实现里面的所有定义过的接口方法:公用接口类里定义了这三个方法 // <<<<<<<<<< 
        get {
            return viewModelProperty.Value;
        }
        set {
            if (!_isInitialized) {
                OnInitialize();
                _isInitialized = true;
            }
            viewModelProperty.Value = value;
        }
    }
    protected virtual void OnInitialize() { // 辅助帮助抽象方法,可以随每个视图里的不同需求再具体定义
        GameObject = ResourceConstant.Loader.LoadClone(BundleName, AssetName, EAssetBundleUnloadLevel.Never);
        GameObject.AddComponent<CanvasGroup>();
        Transform.SetParent(GameObject.Find("ViewRoot").transform, false);
        viewModelProperty.OnValueChanged += OnBindingContextChanged;
    }
// <<<<<<<<<<<<<<<<<<<< 此公用方法关联四个方法:代理模式的RevealedAction + OnAppear() +  OnReveal() + OnRevealed()
// OnAppear() + OnRevealed(): 两个为公用方法,提供给子视图来继承覆写
    public void Reveal(bool immediate = true, Action action = null) { 
        if (action != null) 
            RevealedAction += action;
        OnAppear();
        OnReveal(immediate);
        OnRevealed();
    }
    public void Hide(bool immediate = true, Action action = null) { // <<<<<<<<<<<<<<<<<<<< 
            if (action != null) 
                HiddenAction += action;
        OnHide(immediate);
        OnHidden();
        OnDisappear();
    }
    public virtual void OnAppear() {
        GameObject.SetActive(true);
    }
    private void OnReveal(bool immediate) {
        BindingContext.OnStartReveal();
        if (immediate) {
            Transform.localScale = Vector3.one;
            CanvasGroup.alpha = 1;
        } else 
            StartAnimatedReveal(); // <<<<<<<<<< 
    }
    public virtual void OnRevealed() {
        BindingContext.OnFinishReveal();
        if (RevealedAction != null) 
            RevealedAction();
        if (IsRoot) {
            if (CloseOtherRootView != null) 
                CloseOtherRootView();
        }
        if (SetDownRootIndex != null) 
            SetDownRootIndex();
    }
    private void OnHide(bool immediate) {
        BindingContext.OnStartHide();
        if (immediate) {
            Transform.localScale = Vector3.zero;
            CanvasGroup.alpha = 0;
        } else 
            StartAnimatedHide();
    }
    public virtual void OnHidden() {
        if (HiddenAction != null) 
            HiddenAction();
    }
    public virtual void OnDisappear() {
        GameObject.SetActive(false);
        BindingContext.OnFinishHide();
        if (DestoryOnHide) 
            UnityEngine.Object.Destroy(GameObject);
    }
// OnInitialize() + OnDestory(): ViewModelBase里定义的抽象方法实现,实现必要的基类逻辑    
    public virtual void OnDestory() {
        if (BindingContext.IsRevealed) 
            Hide(true);
        BindingContext.OnDestory();
        BindingContext = null;
        viewModelProperty.OnValueChanged = null;
    }

// 对于视图中需要使用动画的情况:作出了考虑,定义了可以调用的方法    
    protected virtual void StartAnimatedReveal() {
        CanvasGroup.interactable = false;
        Transform.localScale = Vector3.one;
        //huandong
             //CanvasGroup.DOFade(1, 0.2f).SetDelay(0.2f).OnComplete(() =>
             //{
             //    canvasGroup.interactable = true;
             //});
             }
    protected virtual void StartAnimatedHide() {
        CanvasGroup.interactable = false;
        //canvasGroup.DOFade(0, 0.2f).SetDelay(0.2f).OnComplete(() =>
        //{
        //    transform.localScale = Vector3.zero;
        //    canvasGroup.interactable = true;
        //});
    }
    
// 当有用户行为等导致视图变更的时候,需要调用的对所绑定的ViewModel 的变更
    protected virtual void OnBindingContextChanged(ViewModelBase oldValue, ViewModelBase newValue) {
        binder.UnBind(oldValue);
        binder.Bind(newValue);
    }

    // 主要是针对热更新AB(AssetBundle) 包的处理的相关函数的定义
    public virtual string BundleName {
        get {
            return string.Empty;
        }
    }
    public virtual string AssetName {
        get {
            return string.Empty;
        }
    }
    public virtual string ViewName {
        get {
            return string.Empty;
        }
    }
    public virtual string ViewModelTypeName {
        get {
            return string.Empty;
        }
    }
    public GameObject GameObject {
        get;
        set;
    }
    private Transform _transform;
    public Transform Transform {
        get {
            if (_transform == null) {
                _transform = GameObject.transform;
            }
            return _transform;
        }
    }
    private CanvasGroup _canvasGroup;
    public CanvasGroup CanvasGroup {
        get {
            if (_canvasGroup == null) 
                _canvasGroup = GameObject.GetComponent<CanvasGroup>();
            return _canvasGroup;
        }
    }
}
\end{minted}
\subsubsection{UnityGuiViewAdapter : CrossBindingAdapto: 最鼻祖的实体基类}
\label{sec-6-4-3}
\begin{minted}[fontsize=\scriptsize,linenos=false]{csharp}
public class UnityGuiViewAdapter : CrossBindingAdaptor { // 最鼻祖的实体基类

    // 这里是ModuleBaseAdapter里提供的三个接口方法:这里想一想,为什么要实现ModuleBaseAdapter里所定义的三个方法呢,为什么需要如此桥接?
    public override Type BaseCLRType {
        get {
            return typeof(UnityGuiView);
        }
    }
    public override Type AdaptorType {
        get {
            return typeof(UnityGuiViewAdaptor);
        }
    }
    public override object CreateCLRInstance(ILRuntime.Runtime.Enviorment.AppDomain appdomain, ILTypeInstance instance) {
        return new UnityGuiViewAdaptor(appdomain, instance);
    }
    
    class UnityGuiViewAdaptor : UnityGuiView, CrossBindingAdaptorType { // ILRuntime.Enviorment.CrossBindingAdaptorType
        ILTypeInstance instance;
        ILRuntime.Runtime.Enviorment.AppDomain appdomain;
        object[] param2 = new object[2];

        public UnityGuiViewAdaptor() { }
        public UnityGuiViewAdaptor(ILRuntime.Runtime.Enviorment.AppDomain appdomain, ILTypeInstance instance) {
            this.appdomain = appdomain;
            this.instance = instance;
        }
        public ILTypeInstance ILInstance {
            get { return instance; }
        }

// UnityGuiView里所定义的所有公用方法的基类实现:因为后来的继承类可以覆写,但是也可以要求就请按照基类的实现去执行        
        protected override void OnInitialize() { // 辅助帮助抽象方法,可以随每个视图里的不同需求再具体定义
            if (!_onInitializeGot) {
                _onInitialize = instance.Type.GetMethod("OnInitialize");
                _onInitializeGot = true;
            }
            if (_onInitialize != null && !isOnInitializeInvoking) {
                isOnInitializeInvoking = true;
                appdomain.Invoke(_onInitialize, instance);
                isOnInitializeInvoking = false;
            } else 
                base.OnInitialize();
        }
        public override void OnAppear() {
            if (!_onAppearGot) {
                _onAppear = instance.Type.GetMethod("OnAppear");
                _onAppearGot = true;
            }
            if (_onAppear != null && !isOnAppearInvoking) {
                isOnAppearInvoking = true;
                appdomain.Invoke(_onAppear, instance);
                isOnAppearInvoking = false;
            } else {
                base.OnAppear();
            }
        }
        public override void OnRevealed() {
            if (!_onRevealedGot) {
                _onRevealed = instance.Type.GetMethod("OnRevealed");
                _onRevealedGot = true;
            }
            if (_onRevealed != null && !isOnRevealedInvoking) {
                isOnRevealedInvoking = true;
                appdomain.Invoke(_onRevealed, instance);
                isOnRevealedInvoking = false;
            } else {
                base.OnRevealed();
            }
        }
        public override void OnHidden() {
            if (!_onHiddenGot) {
                _onHidden = instance.Type.GetMethod("OnHidden");
                _onHiddenGot = true;
            }
            if (_onHidden != null && !isOnHiddenInvoking) {
                isOnHiddenInvoking = true;
                appdomain.Invoke(_onHidden, instance);
                isOnHiddenInvoking = false;
            } else {
                base.OnHidden();
            }
        }
        public override void OnDisappear() {
            if (!_onDisappearGot) {
                _onDisappear = instance.Type.GetMethod("OnDisappear");
                _onDisappearGot = true;
            }
            if (_onDisappear != null && !isOnDisappearInvoking) {
                isOnDisappearInvoking = true;
                appdomain.Invoke(_onDisappear, instance);
                isOnDisappearInvoking = false;
            } else {
                base.OnDisappear();
            }
        }
        public override void OnDestory() {
            if (!_onDestoryGot) {
                _onDestory = instance.Type.GetMethod("OnDestory");
                _onDestoryGot = true;
            }
            if (_onDestory != null && !isOnDestoryInvoking) {
                isOnDestoryInvoking = true;
                appdomain.Invoke(_onDestory, instance);
                isOnDestoryInvoking = false;
            } else {
                base.OnDestory();
            }
        }
        protected override void StartAnimatedReveal() {
            if (!_startAnimatedRevealGot) {
                _startAnimatedReveal = instance.Type.GetMethod("StartAnimatedReveal");
                _startAnimatedRevealGot = true;
            }
            if (_startAnimatedReveal != null && !isStartAnimatedRevealInvoking) {
                isStartAnimatedRevealInvoking = true;
                appdomain.Invoke(_startAnimatedReveal, instance);
                isStartAnimatedRevealInvoking = false;
            } else {
                base.StartAnimatedReveal();
            }
        }
        protected override void StartAnimatedHide() {
            if (!_startAnimatedHideGot) {
                _startAnimatedHide = instance.Type.GetMethod("StartAnimatedHide");
                _startAnimatedHideGot = true;
            }
            if (_startAnimatedHide != null && !isStartAnimatedHideInvoking) {
                isStartAnimatedHideInvoking = true;
                appdomain.Invoke(_startAnimatedHide, instance);
                isStartAnimatedHideInvoking = false;
            } else {
                base.StartAnimatedHide();
            }
        }
        protected override void OnBindingContextChanged(ViewModelBase oldValue, ViewModelBase newValue) {
            if (!_onBindingContextChangedGot) {
                _onBindingContextChanged = instance.Type.GetMethod("OnBindingContextChanged");
                _onBindingContextChangedGot = true;
            }
            if (_onBindingContextChanged != null && !isOnBindingContextChangedInvoking) {
                isOnBindingContextChangedInvoking = true;
                appdomain.Invoke(_onBindingContextChanged, instance, param2);
                isOnBindingContextChangedInvoking = false;
            } else {
                base.OnBindingContextChanged(oldValue, newValue);
            }
        }

// 下面是处理热更新AB包相关的回调接口控制公用方法:定义为基类实现,因为此类为第一个实体的基类
        public override string BundleName {
            get {
                if (!_getBundleNameGot) {
                    _getBundleName = instance.Type.GetMethod("get_BundleName", 0);
                    _getBundleNameGot = true;
                }
                if (_getBundleName != null && !isGetBundleNameInvoking) {
                    isGetBundleNameInvoking = true;
                    var res = (string)appdomain.Invoke(_getBundleName, instance, null);
                    isGetBundleNameInvoking = false;
                    return res;
                } else {
                    return base.BundleName;
                }
            }
        }
        public override string AssetName {
            get {
                if (!_getAssetNameGot) {
                    _getAssetName = instance.Type.GetMethod("get_AssetName", 0);
                    _getAssetNameGot = true;
                }
                if (_getAssetName != null && !isGetAssetNameInvoking) {
                    isGetAssetNameInvoking = true;
                    var res = (string)appdomain.Invoke(_getAssetName, instance, null);
                    isGetAssetNameInvoking = false;
                    return res;
                } else {
                    return base.AssetName;
                }
            }
        }
        public override string ViewName {
            get {
                if (!_getViewNameGot) {
                    _getViewName = instance.Type.GetMethod("get_ViewName", 0);
                    _getViewNameGot = true;
                }
                if (_getViewName != null && !isGetViewNameInvoking) {
                    isGetViewNameInvoking = true;
                    var res = (string)appdomain.Invoke(_getViewName, instance, null);
                    isGetViewNameInvoking = false;
                    return res;
                } else {
                    return base.ViewName;
                }
            }
        }
        public override string ViewModelTypeName {
            get {
                if (!_getViewModelTypeNameGot) {
                    _getViewModelTypeName = instance.Type.GetMethod("get_ViewModelTypeName", 0);
                    _getViewModelTypeNameGot = true;
                }
                if (_getViewModelTypeName != null && !isGetViewModelTypeNameInvoking) {
                    isGetViewModelTypeNameInvoking = true;
                    var res = (string)appdomain.Invoke(_getViewModelTypeName, instance, null);
                    isGetViewModelTypeNameInvoking = false;
                    return res;
                } else {
                    return base.ViewModelTypeName;
                }
            }
        }

// 覆写UnityGuiView里定义的两个公用抽象方法        
        public override bool DestoryOnHide {
            get {
                if (!_getDestoryOnHideGot) {
                    _getDestoryOnHide = instance.Type.GetMethod("get_DestoryOnHide", 0);
                    _getDestoryOnHideGot = true;
                }
                if (_getDestoryOnHide != null && !isGetDestoryOnHideInvoking) {
                    isGetDestoryOnHideInvoking = true;
                    var res = (bool)appdomain.Invoke(_getDestoryOnHide, instance, null);
                    isGetDestoryOnHideInvoking = false;
                    return res;
                } else {
                    return base.DestoryOnHide;
                }
            }
        }
        public override bool IsRoot {
            get {
                if (!_getIsRootGot) {
                    _getIsRoot = instance.Type.GetMethod("get_IsRoot", 0);
                    _getIsRootGot = true;
                }
                if (_getIsRoot != null && !isGetIsRootInvoking) {
                    isGetIsRootInvoking = true;
                    var res = (bool)appdomain.Invoke(_getIsRoot, instance, null);
                    isGetIsRootInvoking = false;
                    return res;
                } else {
                    return base.IsRoot;
                }
            }
        }
    }
    
// 每个标记变量对应的三小变量    
    IMethod _onInitialize;
    bool _onInitializeGot;
    bool isOnInitializeInvoking = false;

    IMethod _onAppear;
    bool _onAppearGot;
    bool isOnAppearInvoking = false;
    IMethod _onRevealed;
    bool _onRevealedGot;
    bool isOnRevealedInvoking = false;
    IMethod _onHidden;
    bool _onHiddenGot;
    bool isOnHiddenInvoking = false;
    IMethod _onDisappear;
    bool _onDisappearGot;
    bool isOnDisappearInvoking = false;
    IMethod _onDestory;
    bool _onDestoryGot;
    bool isOnDestoryInvoking = false;
    IMethod _startAnimatedReveal;
    bool _startAnimatedRevealGot;
    bool isStartAnimatedRevealInvoking = false;
    IMethod _startAnimatedHide;
    bool _startAnimatedHideGot;
    bool isStartAnimatedHideInvoking = false;
    IMethod _getBundleName;
    bool _getBundleNameGot;
    bool isGetBundleNameInvoking = false;
    IMethod _getAssetName;
    bool _getAssetNameGot;
    bool isGetAssetNameInvoking = false;
    IMethod _getViewName;
    bool _getViewNameGot;
    bool isGetViewNameInvoking = false;
    IMethod _getDestoryOnHide;
    bool _getDestoryOnHideGot;
    bool isGetDestoryOnHideInvoking = false;
    IMethod _getIsRoot;
    bool _getIsRootGot;
    bool isGetIsRootInvoking = false;
    IMethod _getViewModelTypeName;
    bool _getViewModelTypeNameGot;
    bool isGetViewModelTypeNameInvoking = false;
    IMethod _onBindingContextChanged;
    bool _onBindingContextChangedGot;
    bool isOnBindingContextChangedInvoking = false;
}
\end{minted}
\subsection{ViewModel}
\label{sec-6-5}
\subsubsection{ViewModelBase.cs}
\label{sec-6-5-1}
\begin{minted}[fontsize=\scriptsize,linenos=false]{csharp}
public class ViewModelBase {
    private bool _isInitialize;
    public bool IsRevealInProgress {
        get;
        private set;
    }
    public bool IsRevealed {
        get;
        private set;
    }
    public bool IsHideInProgress {
        get;
        private set;
    }
    public ViewModelBase ParentViewModel {
        get;
        set;
    }
    public virtual void OnStartReveal() {
        IsRevealInProgress = true;
        if (!_isInitialize) {
            OnInitialize();
            _isInitialize = true;
        }
    }
    public virtual void OnFinishReveal() {
        IsRevealInProgress = false;
        IsRevealed = true;
    }
    public virtual void OnStartHide() {
        IsHideInProgress = true;
    }
    public virtual void OnFinishHide() {
        IsHideInProgress = false;
        IsRevealed = false;
    }
    
    public virtual void OnDestory() {}
    protected virtual void OnInitialize() {}
}
\end{minted}
\subsubsection{ViewModelBaseAdapter : CrossBindingAdaptor:自定义实现了ViewModel的跨域跨程序集适配}
\label{sec-6-5-2}
\begin{minted}[fontsize=\scriptsize,linenos=false]{csharp}
// 作为两个不同程序集中的ViewModel的桥梁适配器:
public class ViewModelBaseAdapter : CrossBindingAdaptor { // 这里需要再想一想:为什么外面大类,里面小类,继承的基类不一样,公用与私有?

// 实现了基类 CrossBindingAdapter里的其中三个抽象方法
    public override Type BaseCLRType {
        get {
            return typeof(ViewModelBase);
        }
    }
    public override Type AdaptorType {
        get {
            return typeof(ViewModelBaseAdaptor);
        }
    }
    public override object CreateCLRInstance(ILRuntime.Runtime.Enviorment.AppDomain appdomain, ILTypeInstance instance) {
        return new ViewModelBaseAdaptor(appdomain, instance);
    }

// 作为两个不同程序集中的ViewModel的桥梁适配器:继承自ViewModelBase,需要实现里而定义过的甩有6个抽象方法
    class ViewModelBaseAdaptor : ViewModelBase, CrossBindingAdaptorType { // <<<<<<<<<<<<<<<<<<<< 
        ILTypeInstance instance; // 来自于 CrossBindingAdapterType ?
        ILRuntime.Runtime.Enviorment.AppDomain appdomain;

        public ViewModelBaseAdaptor() { }
        public ViewModelBaseAdaptor(ILRuntime.Runtime.Enviorment.AppDomain appdomain, ILTypeInstance instance) {
            this.appdomain = appdomain;
            this.instance = instance;
        }
        
// 来自于 CrossBindingAdapterType接口的实体实现
        public ILTypeInstance ILInstance { get { return instance; } } 
        
// 作为两个不同程序集中的ViewModel的桥梁适配器:继承自ViewModelBase,需要实现里而定义过的甩有6个抽象方法
        public override void OnStartReveal() {
            if (!_onStartRevealGot) {
                _onStartReveal = instance.Type.GetMethod("OnStartReveal");
                _onStartRevealGot = true;
            }
            if (_onStartReveal != null && !_isOnStartRevealInvoking) {
                _isOnStartRevealInvoking = true;
                appdomain.Invoke(_onStartReveal, instance);
                _isOnStartRevealInvoking = false;
            } else 
                base.OnStartReveal();
        }
        public override void OnFinishReveal() {
            if (!_onFinishRevealGot) {
                _onFinishReveal = instance.Type.GetMethod("OnFinishReveal");
                _onFinishRevealGot = true;
            }
            if (_onFinishReveal != null && !_isOnFinishRevealInvoking) {
                _isOnFinishRevealInvoking = true;
                appdomain.Invoke(_onFinishReveal, instance);
                _isOnFinishRevealInvoking = false;
            } else 
                base.OnFinishReveal();
        }
        public override void OnStartHide() {
            if (!_onStartHideGot) {
                _onStartHide = instance.Type.GetMethod("OnStartHide");
                _onStartHideGot = true;
            }
            if (_onStartHide != null && !_isOnStartHideInvoking) {
                _isOnStartHideInvoking = true;
                appdomain.Invoke(_onStartHide, instance);
                _isOnStartHideInvoking = false;
            } else 
                base.OnStartHide();
        }
        public override void OnFinishHide() {
            if (!_onFinishHideGot) {
                _onFinishHide = instance.Type.GetMethod("OnFinishHide");
                _onFinishHideGot = true;
            }
            if (_onFinishHide != null && !_isOnFinishHideInvoking) {
                _isOnFinishHideInvoking = true;
                appdomain.Invoke(_onFinishHide, instance);
                _isOnFinishHideInvoking = false;
            } else 
                base.OnFinishHide();
        }
        public override void OnDestory() {
            if (!_onDestoryGot) {
                _onDestory = instance.Type.GetMethod("OnDestory");
                _onDestoryGot = true;
            }
            if (_onDestory != null && !_isOnDestoryInvoking) {
                _isOnDestoryInvoking = true;
                appdomain.Invoke(_onDestory, instance);
                _isOnDestoryInvoking = false;
            } else 
                base.OnDestory();
        }
        protected override void OnInitialize() {
            if (!_onInitializeGot) {
                _onInitialize = instance.Type.GetMethod("OnInitialize");
                _onInitializeGot = true;
            }
            if (_onInitialize != null && !_isOnInitializeInvoking) {
                _isOnInitializeInvoking = true;
                appdomain.Invoke(_onInitialize, instance);
                _isOnInitializeInvoking = false;
            } else 
                base.OnInitialize();
        }       
    // _onStart/_onFinish: Reveal + Hide;
        IMethod _onStartReveal;
        bool _onStartRevealGot;
        bool _isOnStartRevealInvoking = false;
        IMethod _onFinishReveal;
        bool _onFinishRevealGot;
        bool _isOnFinishRevealInvoking = false;
        IMethod _onStartHide;
        bool _onStartHideGot;
        bool _isOnStartHideInvoking = false;
        IMethod _onFinishHide;
        bool _onFinishHideGot;
        bool _isOnFinishHideInvoking = false;
    // _onInitialize + _onDestroy
        IMethod _onInitialize;
        bool _onInitializeGot;
        bool _isOnInitializeInvoking = false;
        IMethod _onDestory;
        bool _onDestoryGot;
        bool _isOnDestoryInvoking = false;
    }
}
\end{minted}

\section{HotFix中使用MVVM架构实现热更新的搭配与相关的链接}
\label{sec-7}


\section{ILRuntime 类库里源码的基本理解,最重要的涉及到的相关的类与方法摘要}
\label{sec-8}
\subsection{CrossBindingAdaptor : IType 跨域(程序集)绑定适配器 + CrossBindingAdaptorType interface 公用接口类(为什么要这个公用接口?)}
\label{sec-8-1}
\begin{minted}[fontsize=\scriptsize,linenos=false]{csharp}
public interface CrossBindingAdaptorType { // 公用接口类
    ILTypeInstance ILInstance { get; }
}

// This interface is used for inheritance and implementation of CLR Types or interfaces
public abstract class CrossBindingAdaptor : IType {
    IType type;

// 下面是定义的几个公用的抽象方法,供子类实现    
    // This returns the CLR type to be inherited or CLR interface to be implemented
    public abstract Type BaseCLRType { get; }
    // If this Adaptor is capable to impelement multuple interfaces, use this Property, AND BaseCLRType should return null
    public virtual Type[] BaseCLRTypes {
        get {
            return null;
        }
    }
    public abstract Type AdaptorType { get; }
    public abstract object CreateCLRInstance(Enviorment.AppDomain appdomain, ILTypeInstance instance);

    internal IType RuntimeType { get { return type; } set { type = value; } }

// 反射机制的所有可能涉及的相关的方法定义;getters/setters
#region IType Members
    public IMethod GetMethod(string name, int paramCount, bool declaredOnly = false) {
        return type.GetMethod(name, paramCount, declaredOnly);
    }
    public IMethod GetMethod(string name, List<IType> param, IType[] genericArguments, IType returnType = null, bool declaredOnly = false) {
        return type.GetMethod(name, param, genericArguments, returnType, declaredOnly);
    }
    public List<IMethod> GetMethods() {
        return type.GetMethods();
    }
    public int GetFieldIndex(object token) {
        return type.GetFieldIndex(token);
    }
    public IMethod GetConstructor(List<IType> param) {
        return type.GetConstructor(param);
    }
    public bool CanAssignTo(IType type) {
        bool res = false;
        if (BaseType != null)
            res = BaseType.CanAssignTo(type);
        var interfaces = Implements;
        if (!res && interfaces != null) {
            for (int i = 0; i < interfaces.Length; i++) {
                var im = interfaces[i];
                res = im.CanAssignTo(type);
                if (res)
                    return true;
            }
        }
        return res;
    }
    public IType MakeGenericInstance(KeyValuePair<string, IType>[] genericArguments) {
        return type.MakeGenericInstance(genericArguments);
    }
    public IType MakeByRefType() {
        return type.MakeByRefType();
    }
    public IType MakeArrayType(int rank) {
        return type.MakeArrayType(rank);
    }
    public IType FindGenericArgument(string key) {
        return type.FindGenericArgument(key);
    }
    public IType ResolveGenericType(IType contextType) {
        return type.ResolveGenericType(contextType);
    }
    public IMethod GetVirtualMethod(IMethod method) {
        return type.GetVirtualMethod(method);
    }
    public void GetValueTypeSize(out int fieldCout, out int managedCount) {
        type.GetValueTypeSize(out fieldCout, out managedCount);
    }
// Getter / Setter s 
    public bool IsGenericInstance {
        get {
            return type.IsGenericInstance;
        }
    }
    public KeyValuePair<string, IType>[] GenericArguments {
        get {
            return type.GenericArguments;
        }
    }
    public Type TypeForCLR {
        get {
            return type.TypeForCLR;
        }
    }
    public IType ByRefType {
        get {
            return type.ByRefType;
        }
    }
    public IType ArrayType {
        get {
            return type.ArrayType;
        }
    }
    public string FullName {
        get {
            return type.FullName;
        }
    }
    public string Name {
        get {
            return type.Name;
        }
    }
    public bool IsValueType {
        get {
            return type.IsValueType;
        }
    }
    public bool IsPrimitive {
        get {
            return type.IsPrimitive;
        }
    }
    public bool IsEnum {
        get {
            return type.IsEnum;
        }
    }
    public bool IsDelegate {
        get {
            return type.IsDelegate;
        }
    }
    public AppDomain AppDomain {
        get {
            return type.AppDomain;
        }
    }
    public Type ReflectionType {
        get {
            return type.ReflectionType;
        }
    }
    public IType BaseType {
        get {
            return type.BaseType;
        }
    }
    public IType[] Implements {
        get {
            return type.Implements;
        }
    }
    public bool HasGenericParameter {
        get {
            return type.HasGenericParameter;
        }
    }
    public bool IsGenericParameter {
        get {
            return type.IsGenericParameter;
        }
    }
    public bool IsArray {
        get { return false; }
    }
    public bool IsByRef {
        get {
            return type.IsByRef;
        }
    }
    public bool IsInterface {
        get { return type.IsInterface; }
    }
    public IType ElementType {
        get {
            return type.ElementType;
        }
    }
    public int ArrayRank {
        get { return type.ArrayRank; }
    }
    public int TotalFieldCount {
        get {
            return type.TotalFieldCount;
        }
    }
    public StackObject DefaultObject {
        get {
            return default(StackObject);
        }
    }
    public int TypeIndex {
        get {
            return -1;
        }
    }
#endregion
}
\end{minted}
\subsection{}
\label{sec-8-2}
\subsection{}
\label{sec-8-3}
\subsection{}
\label{sec-8-4}
\subsection{}
\label{sec-8-5}
% Emacs 27.1 (Org mode 8.2.7c)
\end{document}