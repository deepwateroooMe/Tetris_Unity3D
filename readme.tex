% Created 2019-07-06 Sat 09:24
\documentclass[9pt, b5paper]{article}
\usepackage[UTF8]{ctex}
\usepackage{fontspec}
\usepackage{graphicx}
\usepackage{xcolor}
\usepackage{multirow}
\usepackage{multicol}
\usepackage{float}
\usepackage{textcomp}
\usepackage{geometry}
\geometry{left=1.2cm,right=1.2cm,top=1.5cm,bottom=1.2cm}
\usepackage{algorithm}
\usepackage{algorithmic}
\usepackage{latexsym}
\usepackage{natbib}
\usepackage{listings}
\usepackage{minted}
\usepackage[xetex,colorlinks=true,CJKbookmarks=true,linkcolor=blue,urlcolor=blue,menucolor=blue]{hyperref}
\author{deepwaterooo}
\date{\today}
\title{Tetris 3D Unity}
\hypersetup{
  pdfkeywords={},
  pdfsubject={},
  pdfcreator={Emacs 26.1 (Org mode 8.2.7c)}}
\begin{document}

\maketitle
\tableofcontents


\section{references}
\label{sec-1}
\subsection{concurrent}
\label{sec-1-1}
\begin{itemize}
\item 用Semaphore实现对象池
\begin{itemize}
\item \url{https://donald-draper.iteye.com/blog/2360817}
\end{itemize}
\begin{minted}[linenos=true]{java}
package juc.latch;  
import java.util.concurrent.Semaphore;  
import java.util.concurrent.locks.Lock;  
import java.util.concurrent.locks.ReentrantLock;  
/** 
 * 信号量实现的对象池 
 * @author donald 
 * 2017年3月6日 
 * 下午9:43:06 
 * @param <T> 
 */

public class ObjectCache<T> {  

    // 对象工厂  
    public interface ObjectFactory<T> {  
        T makeObject();  
    }  

    // 将对象封装节点中,放到一个先进先出的队列中,即对象池  
    class Node {  
        T obj;  
        Node next;  
    }
    
    final int capacity; // 线程次容量  
    final ObjectFactory<T> factory;  
    final Lock lock = new ReentrantLock(); // 保证对象获取,释放的线程安全  
    final Semaphore semaphore; // 信号量  
    private Node head;  
    private Node tail;
    
    public ObjectCache(int capacity, ObjectFactory<T> factory) {  
        this.capacity = capacity;  
        this.factory = factory;  
        this.semaphore = new Semaphore(this.capacity);  
        this.head = null;  
        this.tail = null;  
    }  

    /** 
     * 从对象池中,获取对象 
     * @return 
     * @throws InterruptedException 
     */  
    public T getObject() throws InterruptedException {  
        semaphore.acquire();  
        return getObjectFromPool();  
    }
    
    /** 
     * 线程安全地从对象池获取对象 
     * @return 
     */  
    private T getObjectFromPool() {  
        lock.lock();  
        try {  
            if (head == null) {  
                return factory.makeObject();  
            } else {  
                Node ret = head;  
                head = head.next;  
                if (head == null)  
                tail = null;  
                ret.next = null;//  help GC  
                return ret.obj;  
            }  
        } finally {  
            lock.unlock();  
        }  
    }  
    /** 
     * 线程安全地,将对象放回对象池 
     * @param t 
     */  
    private void putBackObjectToPool(T t) {  
        lock.lock();  
        try {  
            Node node = new Node();  
            node.obj = t;  
            if (tail == null) {  
                head = tail = node;  
            } else {  
                tail.next = node;  
                tail = node;  
            }  
        } finally {  
            lock.unlock();  
        }  
    }  
    /** 
     * 将对象放回对象池 
     * @param t 
     */  
    public void putBackObject(T t) {  
        putBackObjectToPool(t);  
        semaphore.release();  
    }  
}
\end{minted}
\item Object pool pattern
\begin{itemize}
\item \url{https://en.wikipedia.org/wiki/Object_pool_pattern}
\end{itemize}
\begin{minted}[linenos=true]{java}
namespace DesignPattern.Objectpool  {

    // The PooledObject class is the type that is expensive or slow to instantiate,
    // or that has limited availability, so is to be held in the object pool.
    public class PooledObject {
        DateTime _createdAt = DateTime.Now;
        public DateTime CreatedAt {
            get { return _createdAt; }
        }
        public string TempData { get; set; }
    }

    // The Pool class is the most important class in the object pool design pattern. It controls access to the
    // pooled objects, maintaining a list of available objects and a collection of objects that have already been
    // requested from the pool and are still in use. The pool also ensures that objects that have been released
    // are returned to a suitable state, ready for the next time they are requested. 
    public static class Pool {
        private static List<PooledObject> _available = new List<PooledObject>();
        private static List<PooledObject> _inUse = new List<PooledObject>();
        public static PooledObject GetObject() {
            lock(_available) {
                if (_available.Count != 0) {
                    PooledObject po = _available[0];
                    _inUse.Add(po);
                    _available.RemoveAt(0);
                    return po;
                } else {
                    PooledObject po = new PooledObject();
                    _inUse.Add(po);
                    return po;
                }
            }
        }
        public static void ReleaseObject(PooledObject po) {
            CleanUp(po);
            lock (_available) {
                _available.Add(po);
                _inUse.Remove(po);
            }
        }
        private static void CleanUp(PooledObject po) {
            po.TempData = null;
        }
    }
}
\end{minted}

\item Sun‘刺眼的博客: 随笔分类 - Unity3D、C\#
\begin{itemize}
\item \url{https://www.cnblogs.com/android-blogs/category/879304.html}
\end{itemize}
\item Unity协程(Coroutine)原理深入剖析
\begin{itemize}
\item \url{https://dsqiu.iteye.com/blog/2029701}
\end{itemize}
\item Unity3d IEnumerator 协程的理解
\begin{itemize}
\item \url{https://blog.csdn.net/jasonwang18/article/details/55519165}
\end{itemize}
\item 关于对象池的一些分析
\begin{itemize}
\item \url{https://droidyue.com/blog/2016/12/12/dive-into-object-pool/}
\end{itemize}
\end{itemize}

\subsection{Э³Ì Coroutine}
\label{sec-1-2}
\begin{itemize}
\item \url{http://dsqiu.iteye.com/blog/2029701}
\item \url{http://dsqiu.iteye.com/blog/2049743}
\end{itemize}

\subsection{tetris 3d specific}
\label{sec-1-3}
\begin{itemize}
\item \url{https://www.youtube.com/watch?v=UZSotPFf0ug} with tutorial, Maya Unity
\item above 2d tutorial \url{http://noobtuts.com/unity/2d-tetris-game}
\item commands \url{http://users.csc.calpoly.edu/~zwood/teaching/csc471/finalproj24/gzipkin/}
\item 3 other resources: 
\begin{itemize}
\item \url{http://subject.manew.com/source/index.html}
\item \url{http://jingyan.baidu.com/article/4e5b3e195bde8991901e243a.html}
\item \url{http://www.cnblogs.com/bitzhuwei/p/unity3d-tank-sniper.html}
\end{itemize}
\end{itemize}
\subsection{buttons}
\label{sec-1-4}
\begin{itemize}
\item \url{https://forum.unity3d.com/threads/touch-and-hold-a-button-on-new-ui.266065/}
\item \url{https://stackoverflow.com/questions/38198745/how-to-detect-left-mouse-click-but-not-when-the-click-occur-on-a-ui-button-compo}
\end{itemize}
\subsection{3d games}
\label{sec-1-5}
\begin{itemize}
\item \url{https://www.youtube.com/watch?v=_oEUJ_sirC8} with vedio downloaded
\item 
\item 
\item 
\item 
\item 
\item 
\item 
\end{itemize}
% Emacs 26.1 (Org mode 8.2.7c)
\end{document}