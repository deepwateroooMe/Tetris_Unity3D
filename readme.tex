% Created 2022-09-29 Thu 22:07
\documentclass[9pt, b5paper]{article}
\usepackage{xeCJK}
\usepackage{minted}
\usepackage[T1]{fontenc}
\usepackage[scaled]{beraserif}
\usepackage[scaled]{berasans}
\usepackage[scaled]{beramono}
\usepackage{graphicx}
\usepackage{xcolor}
\usepackage{multirow}
\usepackage{multicol}
\usepackage{float}
\usepackage{textcomp}
\usepackage{algorithm}
\usepackage{algorithmic}
\usepackage{latexsym}
\usepackage{natbib}
\usepackage{geometry}
\geometry{left=1.2cm,right=1.2cm,top=1.5cm,bottom=1.2cm}
\newminted{common-lisp}{fontsize=\footnotesize} 
\usepackage[xetex,colorlinks=true,CJKbookmarks=true,linkcolor=blue,urlcolor=blue,menucolor=blue]{hyperref}
\author{deepwaterooo}
\date{\today}
\title{Unity Android SDK/NDK 俄罗斯方块砖3D小游戏}
\hypersetup{
  pdfkeywords={},
  pdfsubject={},
  pdfcreator={Emacs 27.1 (Org mode 8.2.7c)}}
\begin{document}

\maketitle
\tableofcontents


\section{游戏基础框架的快速链接}
\label{sec-1}
\begin{itemize}
\item 将游戏的绝大部分逻辑放到热更新程序包下执行,以便实现随时热更;
\item 暂时不考虑服务器,所以先弄个本地服务器试运行着
\item 把基本的unity游戏逻辑与ILRuntime + MVVM的唯一一个(如果我需要多个场景,那么我需要添加很多场景切换的逻辑,所以最简单的办法就是游戏自始自终只使用一个场景,其它所有变更的都只是Panel View UI。)场景试着尽快运行起来
\item 打包第一个场景: 好像是两个前后相差几个的游戏引擎不太兼容,打包的预设不识别,会再看一看(先粗糙地封了一整个场景,会把可能热更新的折下来折小一点儿,晚些时候)
\end{itemize}

\includegraphics[width=.9\linewidth]{./pic/readme_20220929_220207.png}
\begin{itemize}
\item 
\item 
\end{itemize}
\section{把原理弄懂}
\label{sec-2}
\begin{itemize}
\item 热更新模块的实充:以前的设计模式和实现的功能还是比较完整的;现在更成熟一点儿,需要把热更新模块补充出来;
\item ILRuntime + MVVM框架设计:两者结合,前几年的时候没能把MVVM理解透彻;
\item 上次前几年主要的难点:好像是在把MVVM双向数据绑定理解得不透彻;那么这次应该就狠没有问题了,更该寻求更好的设计与解决方案
\item 性能优化:另外是对其实高级开发的越来越熟悉,希望应用的性能表现,尤其是渲染性能与速度等、这些更为高级和深入的特性成为这次二次开发的重点。

\item 现在是把自己几年前的写的游戏全忘记了,需要回去把自己的源码找出来,再读一读熟悉一下自己的源码,了解当时设计的估缺点,由此改进更将
\end{itemize}

\section{环境弄得比较好的包括:}
\label{sec-3}
\begin{itemize}
\item 电脑的配置有限,文件稍微大一点儿的时候已经不太好处理了;所以不得不分割成多个小文件
\item 几年过去了,ILRuntime已经不是最新最前沿的热更新技术,成为别人更新技术的一个子模块,所以还是自己再搜索找一下有没有更方便的热更新实现方法(若是不得,我就在自己游戏里实现 ILRuntime + MVVM实现视图等的更新)
\end{itemize}
- 这一两天作必要的文献研究,确定哪个大的模块版块需要实现或是修改优化,列个大致计划,把它们一一完成;希望截止这个周末周六周日能够把这个部分确定得相对精确
\begin{itemize}
\item 小笔记本电脑太慢了,会回家再读其它模块的源码,理解透彻。爱表哥,爱生活!!
\item 输入法的搭建:终于用到了自己之前用过的好用的输入法
\item 这两天开车疲累,最迟明天中午会去南湾找房间出租,尽快解决搬家的问题;昨天晚上回来得太晚了,一路辛苦,路上只差睡着,回到家里补觉补了好多个小时。
\item 小电脑,笔记本电脑里的游戏环境搭建,今天下午去图书馆里弄(今天下午去图书馆里把需要借助快速网络来完成的事情都搭建好;家里被恶房东故意整了个腾腾慢的网,故意阻碍别人的发展,谁还愿意再这样的环境中继续住下去呢?!!!)
\end{itemize}
- 能够把程序源码读得比较懂,也并不代表把所有相关的原理就全部弄懂了;不是说还有多在的挑战,而是说要不断寻找更为有效的学习方法,快速掌握所有涉及到的相关原理;在理解得更为深入掌握了基本原理的基础上再去读源码,会不会更为有效事半功倍呢?这是一颗永远不屈服的心,爱表哥,爱生活!!!
\section{ILRuntime 库的系统再深入理解}
\label{sec-4}
\subsection{ILRuntime基本原理}
\label{sec-4-1}
\begin{itemize}
\item ILRuntime借助Mono.Cecil库来读取DLL的PE信息,以及当中类型的所有信息,最终得到方法的IL汇编码,然后通过内置的IL解译执行虚拟机来执行DLL中的代码。IL解释器代码在ILIntepreter.cs,通过Opcode来逐语句执行机器码,解释器的代码有四千多行。
\end{itemize}

\includegraphics[width=.9\linewidth]{./pic/readme_20220926_094936.png}

\subsection{ILRuntime热更流程}
\label{sec-4-2}

\includegraphics[width=.9\linewidth]{./pic/readme_20220926_095022.png}
\subsection{ILRuntime主要限制}
\label{sec-4-3}

\includegraphics[width=.9\linewidth]{./pic/readme_20220926_095555.png}
\begin{itemize}
\item \textbf{委托适配器(DelegateAdapter)} :将委托实例传出给ILRuntime外部使用,将其转换成CLR委托实例。
\end{itemize}
由于IL2CPP之类的AOT编译技术无法在运行时生成新的类型,所以在创建委托实例的时候ILRuntime选择了显式注册的方式,以保证问题不被隐藏到上线后才发现。
\begin{minted}[fontsize=\scriptsize,linenos=false]{csharp}
//同一参数组合只需要注册一次
delegate void SomeDelegate(int a, float b);
Action<int, float> act;
//注册,不带返回值,最多支持五个参数传入
appDomain.DelegateManager.RegisterMethodDelegate<int, float>();

//注册,带参数返回值,最后一个参数为返回值,最多支持四个参数传入
delegate bool SomeFunction(int a, float b);
Func<int, float, bool> act;
\end{minted}
\begin{itemize}
\item \textbf{委托转换器RegisterDelegateConvertor} :需要将一个不是Action或者Func类型的委托实例传到ILRuntime外部使用,需要写委托适配器和委托转换器。委托转换器将Action和Func转换成你真正需要的那个委托类型
\end{itemize}
\begin{minted}[fontsize=\scriptsize,linenos=false]{csharp}
app.DelegateManager.RegisterDelegateConvertor<SomeFunction>((action) =>
{
    return new SomeFunction((a, b) =>
    {
       return ((Func<int, float, bool>)action)(a, b);
    });
});
\end{minted}
\begin{itemize}
\item 为了避免不必要的麻烦,以及后期热更出现问题,建议: 1、尽量避免不必要的跨域委托调用 2、尽量使用Action以及Func委托类型
\item \textbf{CLR重定向:} ILRuntime为了解决外部调用内部接口的问题,引入了CLR重定向机制。 原理就是当IL解译器发现需要调用某个指定CLR方法时,将实际调用重定向到另外一个方法进行挟持,再在这个方法中对ILRuntime的反射的用法进行处理
\item 从代码中可以看出重定向的工作是把方法挟持下来后装到ILIntepreter的解释器里面实例化
\item 不带返回值的重定向:
\end{itemize}
\begin{minted}[fontsize=\scriptsize,linenos=false]{csharp}
public static StackObject* CreateInstance(ILIntepreter intp, StackObject* esp,
                                          List<object> mStack, CLRMethod method, bool isNewObj) {
    // 获取泛型参数<T>的实际类型
    IType[] genericArguments = method.GenericArguments;
    if (genericArguments != null && genericArguments.Length == 1) {
        var t = genericArguments[0];
        if (t is ILType) { // 如果T是热更DLL里的类型 
            // 通过ILRuntime的接口来创建实例
            return ILIntepreter.PushObject(esp, mStack, ((ILType)t).Instantiate());
        } else // 通过系统反射接口创建实例
            return ILIntepreter.PushObject(esp, mStack, Activator.CreateInstance(t.TypeForCLR));
    } else
        throw new EntryPointNotFoundException();
}
// 注册
foreach (var i in typeof(System.Activator).GetMethods()) {
    // 找到名字为CreateInstance,并且是泛型方法的方法定义
    if (i.Name == "CreateInstance" && i.IsGenericMethodDefinition) {
        // RegisterCLRMethodRedirection:通过redirectMap存储键值对MethodBase-CLRRedirectionDelegate,如果i不为空且redirectMap中没有传入的MethodBase(即下方的i)则存储redirectMap[i] = CreateInstance。所以如此看来注册行为就是把键值对存储到redirectMap的过程
        appdomain.RegisterCLRMethodRedirection(i, CreateInstance);
    }
}
\end{minted}
\begin{itemize}
\item 带返回值方法的重定向
\end{itemize}
\begin{minted}[fontsize=\scriptsize,linenos=false]{csharp}
public unsafe static StackObject* DLog(ILIntepreter __intp, StackObject* __esp,
                                       List<object> __mStack, CLRMethod __method, bool isNewObj)  {
    ILRuntime.Runtime.Enviorment.AppDomain __domain = __intp.AppDomain;
    StackObject* ptr_of_this_method;
    // 只有一个参数,所以返回指针就是当前栈指针ESP - 1
    StackObject* __ret = ILIntepreter.Minus(__esp, 1);
    // 第一个参数为ESP -1, 第二个参数为ESP - 2,以此类推
    ptr_of_this_method = ILIntepreter.Minus(__esp, 1);
    // 获取参数message的值
    object message = StackObject.ToObject(ptr_of_this_method, __domain, __mStack);
    // 需要清理堆栈
    __intp.Free(ptr_of_this_method);
    // 如果参数类型是基础类型,例如int,可以直接通过int param = ptr_of_this_method->Value获取值,
    // 关于具体原理和其他基础类型如何获取,请参考ILRuntime实现原理的文档。
            
    // 通过ILRuntime的Debug接口获取调用热更DLL的堆栈
    string stackTrace = __domain.DebugService.GetStackTrance(__intp);
    Debug.Log(string.Format("{0}\n{1}", format, stackTrace));
    return __ret;
}
\end{minted}
\begin{itemize}
\item \textbf{LitJson集成}: Json序列化是开发中非常经常需要用到的功能,考虑到其通用性,因此ILRuntime对LitJson这个序列化库进行了集成
\end{itemize}
\begin{minted}[fontsize=\scriptsize,linenos=false]{csharp}
//对LitJson进行注册,需要在注册CLR绑定之前
LitJson.JsonMapper.RegisterILRuntimeCLRRedirection(appdomain);
//LitJson使用
//将一个对象转换成json字符串
string json = JsonMapper.ToJson(obj);
//json字符串反序列化成对象
JsonTestClass obj = JsonMapper.ToObject<JsonTestClass>(json);
\end{minted}
\begin{itemize}
\item \textbf{ILRuntime的性能优化}
\begin{itemize}
\item 值类型优化:使用ILRuntime外部定义的值类型(例如UnityEngine.Vector3)在默认情况下会造成额外的装箱拆箱开销。ILRuntime在1.3.0版中增加了值类型绑定(ValueTypeBinding)机制,通过对这些值类型添加绑定器,可以大幅增加值类型的执行效率,以及避免GC Alloc内存分配。
\item 大规模数值计算:如果在热更内需要进行大规模数值计算,则可以开启ILRuntime在2.0版中加入的寄存器模式来进行优化
\item 避免使用foreach:尽量避免使用foreach,会不可避免地产生GC。而for循环不会。
\item 加载dll并在逻辑后处理进行简单调用
\item 整个文件流程:创建IEnumerator并运行->用文件流判断并读入dll和pdb->尝试加载程序集dll->(如果加载成功)初始化脚本引擎(InitializeILRuntime())->执行脚本引擎加载后的逻辑处理(OnHotFixLoaded())->程序销毁(在OnDestoy中关闭dll和pdb的文件流)
\item MemoryStream:为系统提供流式读写。MemoryStream类封装一个字节数组,在构造实例时可以使用一个字节数组作为参数,但是数组的长度无法调整。使用默认无参数构造函数创建实例,可以使用Write方法写入,随着字节数据的写入,数组的大小自动调整。 参考博客:传送门
\item appdomain.LoadAssembly:将需要热更的dll加载到解释器中。第一个填入dll以及pdb,这里的pdb应该是dll对应的一些标志符号。 后面的ILRuntime.Mono.Cecil.Pdb.PdbReaderProvider()是动态修改程序集,它的作用是给ILRuntime.Mono.Cecil.Pdb.PdbReaderProvider()里的GetSymbolReader)(传入两个参数,一个是通过转化后的ModuleDefinition.ReadModule(stream(即dll))模块定义,以及原来的symbol(即pdb) GetSymbolReader主要的作用是检测其中的一些符号和标志是否为空,不为空的话就进行读取操作。 (这些内容都是ILRuntime中的文件来完成)
\end{itemize}
\item Unity MonoBehaviour lifecycle methods callback execute orders:
\item 还有一个看起来不怎么清楚的,将就凑合着看一下:这几个图因为文件地址错误丢了,改天再补一下
\item IL热更优点:
\begin{itemize}
\item 1、无缝访问C\#工程的现成代码,无需额外抽象脚本API
\item 2、直接使用VS2015进行开发,ILRuntime的解译引擎支持.Net 4.6编译的DLL
\item 3、执行效率是L\#的10-20倍
\item 4、 \textbf{选择性的CLR绑定使跨域调用更快速,绑定后跨域调用的性能能达到slua的2倍左右(从脚本调用GameObject之类的接口)}
\item 5、支持跨域继承(代码里的完美学演示)
\item 6、完整的泛型支持(代码里的完美学演示)
\item 7、拥有Visual Studio的调试插件,可以实现真机源码级调试。支持Visual Studio 2015 Update3 以及Visual Studio 2017和Visual Studio 2019
\item 8、最新的2.0版引入的寄存器模式将数学运算性能进行了大幅优化
\end{itemize}
\end{itemize}

\subsection{ILRuntime启动调试}
\label{sec-4-4}
\begin{itemize}
\item ILRuntime建议全局只创建一个AppDomain,在函数入口添加代码启动调试服务
\end{itemize}
\begin{minted}[fontsize=\scriptsize,linenos=false]{csharp}
appdomain.DebugService.StartDebugService(56000)
\end{minted}
\begin{itemize}
\item 运行主工程(Unity工程)
\item 在热更的VS工程中 点击 - 调试 - 附加到ILRuntime调试,注意使用一样的端口
\item 如果使用VS2015的话需要Visual Studio 2015 Update3以上版本
\end{itemize}
\subsection{线上项目和资料}
\label{sec-4-5}
\begin{itemize}
\item 掌趣很多项目都是使用ILRuntime开发,并上线运营,比如:真红之刃,境·界 灵压对决,全民奇迹2,龙族世界,热血足球
\item 初音未来:梦幻歌姬 使用补丁方式:\url{https://github.com/wuxiongbin/XIL}
\item 本文流程图摘自:ILRuntime的QQ群的《ILRuntime热更框架.docx》(by a 704757217)
\item Unity实现c\#热更新方案探究(三): \url{https://zhuanlan.zhihu.com/p/37375372}
\end{itemize}
% Emacs 27.1 (Org mode 8.2.7c)
\end{document}